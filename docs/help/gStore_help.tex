\documentclass[titlepage, a4paper, 12pt]{article}

%\usepackage{ctex}
\usepackage{lmodern}
\usepackage{ifxetex,ifluatex}
\usepackage{fixltx2e} 
\usepackage{amsmath}
\usepackage{txfonts}
\usepackage{amssymb}
\usepackage{times}
\usepackage{graphicx}
\usepackage{epsfig,tabularx,amssymb,amsmath,subfigure,multirow}
%\usepackage{algorithmic}
\usepackage[linesnumbered,ruled,noend]{algorithm2e}
\usepackage[noend]{algorithmic}
\usepackage{multirow}
\usepackage{graphicx,floatrow}
\usepackage{listings}
\usepackage{threeparttable}
%\usepackage{tikz}
\usepackage[T1]{fontenc}
\usepackage{pgfplots}
\usepackage{filecontents}
\usepackage{comment}

\lstset{%
	alsolanguage=Java,
	%language={[ISO]C++},       %language为,还有{[Visual]C++}
	%alsolanguage=[ANSI]C,      %可以添加很多个alsolanguage,如alsolanguage=matlab,alsolanguage=VHDL等
	%alsolanguage= tcl,
	alsolanguage= XML,
	tabsize=4, %
	frame=shadowbox, %把代码用带有阴影的框圈起来
	commentstyle=\color{red!50!green!50!blue!50},%浅灰色的注释
	rulesepcolor=\color{red!20!green!20!blue!20},%代码块边框为淡青色
	keywordstyle=\color{blue!90}\bfseries, %代码关键字的颜色为蓝色,粗体
	showstringspaces=false,%不显示代码字符串中间的空格标记
	stringstyle=\ttfamily, % 代码字符串的特殊格式
	keepspaces=true, %
	breakindent=22pt, %
	numbers=left,%左侧显示行号 往左靠,还可以为right,或none,即不加行号
	stepnumber=1,%若设置为2,则显示行号为1,3,5,即stepnumber为公差,默认stepnumber=1
	%numberstyle=\tiny, %行号字体用小号
	numberstyle={\color[RGB]{0,192,192}\tiny} ,%设置行号的大小,大小有tiny,scriptsize,footnotesize,small,normalsize,large等
	numbersep=8pt,  %设置行号与代码的距离,默认是5pt
	basicstyle=\footnotesize, % 这句设置代码的大小
	showspaces=false, %
	flexiblecolumns=true, %
	breaklines=true, %对过长的代码自动换行
	breakautoindent=true,%
	breakindent=4em, %
	%	escapebegin=\begin{CJK*}{GBK}{hei},escapeend=\end{CJK*},
	aboveskip=1em, %代码块边框
	tabsize=2,
	showstringspaces=false, %不显示字符串中的空格
	backgroundcolor=\color[RGB]{245,245,244},   %代码背景色
	%backgroundcolor=\color[rgb]{0.91,0.91,0.91}    %添加背景色
	escapeinside=``,  %在``里显示中文
	%% added by http://bbs.ctex.org/viewthread.php?tid=53451
	fontadjust,
	captionpos=t,
	framextopmargin=2pt,framexbottommargin=2pt,abovecaptionskip=-3pt,belowcaptionskip=3pt,
	xleftmargin=4em,xrightmargin=4em, % 设定listing左右的空白
	texcl=true,
	% 设定中文冲突,断行,列模式,数学环境输入,listing数字的样式
	extendedchars=false,columns=flexible,mathescape=true
	% numbersep=-1em
}

\ifnum 0\ifxetex 1\fi\ifluatex 1\fi=0 % if pdftex
  \usepackage[T1]{fontenc}
  \usepackage[utf8]{inputenc}
\else % if luatex or xelatex
  \ifxetex
    \usepackage{mathspec}
    \usepackage{xltxtra,xunicode}
  \else
    \usepackage{fontspec}
  \fi
  \defaultfontfeatures{Mapping=tex-text,Scale=MatchLowercase}
  \newcommand{\euro}{�}
\fi

% use upquote if available, for straight quotes in verbatim environments
\IfFileExists{upquote.sty}{\usepackage{upquote}}{}
% use microtype if available
\IfFileExists{microtype.sty}{%
\usepackage{microtype}
\UseMicrotypeSet[protrusion]{basicmath} % disable protrusion for tt fonts
}{}
\usepackage{longtable,booktabs}
\ifxetex
  \usepackage[setpagesize=false, % page size defined by xetex
              unicode=false, % unicode breaks when used with xetex
              xetex]{hyperref}
\else
  \usepackage[unicode=true]{hyperref}
\fi
\hypersetup{breaklinks=true,
            bookmarks=true,
            pdfauthor={},
            pdftitle={Gstore System},
            colorlinks=true,
            citecolor=blue,
            urlcolor=blue,
            linkcolor=magenta,
            pdfborder={0 0 0}}
\urlstyle{same}  % don't use monospace font for urls
%\setlength{\parskip}{6pt plus 2pt minus 1pt}
\setlength{\emergencystretch}{3em}  % prevent overfull lines
\setcounter{secnumdepth}{0}
\setlength{\parindent}{0pt}
%\setlength{\parindent}{2em}
\addtolength{\parskip}{3pt}
\linespread{1.3}

\begin{document}
\title{\includegraphics[scale=0.3, bb=0 0 385 567]{logo.png} \\
	The handbook of gStore System测试}
%\author{Bookug Lobert\footnote{EECS of Peking University, zengli-bookug@pku.edu.cn}\\[2ex]}
\author{Edited by gStore team \footnote{The mailing list is given in Chapter 13.}}
\date{\today}
%\begin{figure}[b]
%	\centering
% 	\includegraphics[scale=0.3,bb=0 0 385 567]{../logo.png}
	%\caption{Some description about the picture}
%	\label{logo}
%\end{figure}
\maketitle

\hyperdef{}{MathJaxux5fSVGux5fHidden}{}

\hyperdef{}{wmd-preview}{}

\setcounter{tocdepth}{4}
\tableofcontents
\clearpage

\section{Preface}
The RDF (\emph{R}esource \emph{D}escription \emph{F}ramework) is a family of specifications proposed by W3C for modeling Web objects as part of developing the semantic web. In RDF model, each Web object is modeled as a uniquely named \emph{resource} and denoted by a URI (\emph{U}niform \emph{R}esource \emph{I}dentifier). RDF also uses URIs to name the properties of resources and the relationships between resources as well as the two ends of the link (this is usually referred to as a ``triple''). Hence, an RDF dataset can be represented as a directed, labeled graph where resources are vertices, and triples are
edges with property or relationship names as edge labels. For more details, please go to \href{https://www.w3.org/RDF/}{RDF Introduction}\\

To retrieve and manipulate an RDF graph, W3C also proposes a structured query language, SPARQL (\emph{S}imple \emph{P}rotocol \emph{A}nd \emph{R}DF \emph{Q}uery \emph{L}anguage), to access RDF repository. SPARQL contains capabilities for querying required and optional graph patterns along with their conjunctions and disjunctions. SPARQL also supports aggregation, subqueries, negation, creating values by expressions, extensible value testing, and constraining queries by source RDF graph. Similar to RDF graphs, a SPARQL query can also be modeled as a graph, which is a query graph with some variables. Then, evaluating a SPARQL query is equivalent to finding subgraph (homomorphism) matches of a query graph over an RDF graph. You can have a better understanding of SPARQL at \href{https://www.w3.org/TR/sparql11-query/}{SPARQL Introduction}.\\

Although there are some RDF data management systems (like Jena, Virtuoso, Sesame) that store the RDF data in relational systems, few existing systems exploit the native graph pattern
matching semantics of SPARQL. \textbf{Here, we implement a graph-based RDF triple store named gStore, which is a joint research project by Peking University, University of Waterloo and Hong Kong University of Science and Technology. The system is developed and maintained by the database group in Institute of Computer Science and Technology, Peking University, China.} A detailed description of gStore can be found at our papers {[}Zou et al., VLDB 11{]} and {[}Zou et al., VLDB Journal 14{]} in the \hyperref[chapter09]{Publication} chapter. This HELP document includes system installment, usage, API, use cases and FAQ. gStore is a open-source project in github under the BSD license. You are welcome to use gStore, report bugs or suggestions, or join us to make gStore better. It is also allowed for you to build all kinds of applications based on gStore, while respecting our work.\\



\textbf{Please make sure that you have read \hyperref[chapter18]{Legal Issues} before using gStore.}

\clearpage

\part{Start}

\hyperdef{}{chapter00}{\subsection{Chapter 00: A Quick Tour}\label{chapter00}}
Gstore System(also called gStore) is a graph database engine for managing large graph-structured data, which is open-source and targets at Linux operation systems. The whole project is written in C++, with the help of some libraries such as readline, antlr, and so on. Only source tarballs are provided currently, which means you have to compile the source code if you want to use our system.

\hyperdef{}{getting-started}{\subsubsection{Getting
Started}\label{getting-started}}

This system is really user-friendly and you can pick it up in several minutes. Remember to check your platform where you want to run this system by viewing \hyperref[chapter01]{System Requirements}. After all are verified, please get this project's source code. There are several ways to do this:

\begin{itemize}
\item
  download the zip from this repository and extract it
\item
  fork this repository in your github account
\item
  type \texttt{git\ clone\ git@github.com:Caesar11/gStore.git} in your
  terminal or use git GUI to acquire it
\end{itemize}

Then you need to compile the project, just type \texttt{make} in the gStore root directory, and all executables will be ok. To run gStore, please type \texttt{bin/gbuild\ database\_name\ dataset\_path} to build a database named by yourself. And you can use \texttt{bin/gquery\ database\_name} command to query a existing database. What is more, \texttt{bin/gconsole} is a wonderful tool designed for you, providing all operations you need to use gStore.
Notice that all commands should be typed in the root directory of gStore.

\emph{A detailed description can be found at Chapter 04
\hyperref[chapter04]{How to use} in this document.}

\hyperdef{}{advanced-help}{\subsubsection{Advanced
Help}\label{advanced-help}}

If you want to understand the details of the gStore system, or you want to try some advanced operations(for example, using the API, server/client), please see the chapters below.

\begin{itemize}
\item
  \hyperref[chapter02]{Basic Introduction}: introduce the theory and features of gStore
\item
  \hyperref[chapter03]{Install Guide}: instructions on how to install this system
\item
  \hyperref[chapter04]{How To Use}: detailed information about using the gStore system
\item
  \hyperref[chapter05]{Socket API Explanation}: guide you to develop applications based on our Socket API
\item  
  \hyperref[chapter06]{HTTP API Explanation}: guide you to develop applications based on our HTTP API
\item
  \hyperref[chapter08]{Project Structure}: show the whole structure and sequence of this project
\item
  \hyperref[chapter09]{Publications}: contain essays and publications
  related with gStore
\item
  \hyperref[chapter10]{Update Logs}: keep the logs of the system updates
\item
  \hyperref[chapter15]{Test Result}: present the test results of a series of experiments
\end{itemize}

\hyperdef{}{other-business}{\subsubsection{Other Business}\label{other-business}}

We have written a series of short essays addressing recurring challenges in using gStore to realize applications, which are placed in
\hyperref[chapter12]{Recipe Book}.

You are welcome to report any advice or errors in the github Issues part of this repository, if not requiring in-time reply. However, if you want to urgent on us to deal with your reports, please email to to submit your suggestions and report bugs to us by emailing to . A full list of our whole team is in \hyperref[chapter13]{Contributors}.

There are some restrictions when you use the current gStore project, you can see them on \hyperref[chapter10]{Limitations}.

Sometimes you may find some strange phenomena(but not wrong case), or something hard to understand/solve(don't know how to do next), then do not hesitate to visit the \hyperref[chapter11]{Frequently Asked Questions} page.

Graph database engine is a new area and we are still trying to go further. Things we plan to do next is in \hyperref[chapter16]{Future Plan} chapter, and we hope more and more people will support or even
join us. You can support in many ways:

\begin{itemize}
\item
  watch/star our project
\item
  fork this repository and submit pull requests to us
\item
  download and use this system, report bugs or suggestions
\item
  \ldots{}
\end{itemize}

People who inspire us or contribute to this project will be listed in the \hyperref[chapter17]{Thanks List} chapter.

\clearpage

\hyperdef{}{chapter01}{\subsection{Chapter 01: System Requirements}\label{chapter01}}

\emph{We have tested on linux server with CentOS 6.2 x86\_64 and CentOS 6.6 x86\_64. The version of GCC should be 4.4.7 or later.}

\begin{longtable}[c]{@{}ll@{}}
\toprule
Item & Requirement\tabularnewline
\midrule
\endhead
operation system & Linux, such as CentOS, Ubuntu and so on\tabularnewline
architecture & x86\_64\tabularnewline
disk size & according to size of dataset\tabularnewline
memory size & according to size of dataset\tabularnewline
glibc & version \textgreater{}= 2.14\tabularnewline
gcc & version \textgreater{}= 4.4.7\tabularnewline
g++ & version \textgreater{}= 4.4.7\tabularnewline
make & need to be installed\tabularnewline
readline & need to be installed\tabularnewline
readline-devel & need to be installed\tabularnewline
openjdk & needed if using Java api\tabularnewline
openjdk-devel & needed if using Java api\tabularnewline
realpath & needed if using gconsole\tabularnewline
ccache & optional, used to speed up the compilation\tabularnewline
\bottomrule
\caption{software requirement}
\end{longtable}

NOTICE:

\begin{enumerate}
\item
  The name of some packages may be different in different platforms, just install the corresponding one in your own operation system.
\item
  To install readline and readline-devel, just type \texttt{dnf\ install\ readline-devel} in Redhat/CentOS/Fedora, or \texttt{apt-get\ install\ libreadline-dev} in Debian/Ubuntu. Please use corresponding commands in other systems. If you use ArchLinux, just type \texttt{pacman\ -S\ readline} to install the readline and readline-devel.(so do other packages)
\item
  You do not have to install realpath to use gStore, but if you want to use the gconsole for its convenience, please do so by using \texttt{dnf\ install\ realpath} or \texttt{apt-get\ install\ realpath}.
\item
  Our programs use regEx functions, which are provided by GNU/Linux by default. You do not need to have to install boost and boost-devel for more powerful regEx libraries.
\item
  ANTLR3.4 is used in gStore to produce lexer and parser code for SPARQL query. However, you do not need to install the corresponding antlr libraries because we have merged the libantlr3.4 in our system.
\item
  When you type \texttt{make} in the root directory of the gStore project, the Java api will also be compiled. You can modify the makefile if you do not have JDK in your system. However, you are advised to install openjdk-devel in your Linux system.
\item
  To install ccache, you need to add epel repository if using CentOS, while in Ubuntu you can directly install it by 'apt-get install ccache' comand. If you can not install ccahe(or maybe you do not want to), please go to modify the makefile(just change the CC variable to g++).
\item
  Any other questions, please go to \hyperref[chapter11]{FAQ} page.
\end{enumerate}

\clearpage

\hyperdef{}{chapter02}{\subsection{Chapter 02: Basic Introduction}\label{chapter02}}

\textit{The first essay to come up with Gstore System is
\href{run:../pdf/gStoreVLDBJ.pdf}{gStore\_VLDBJ}, and you can find related publications in
\hyperref[chapter09]{Publications}.}

\hyperdef{}{what-is-gstore}{\subsubsection{What Is
gStore}\label{what-is-gstore}}

gStore is a graph-based RDF data management system(or what is commonly called a ``triple store'') that maintains the graph structure of the original \href{http://www.w3.org/TR/rdf11-concepts/}{RDF} data. Its data model is a labeled, directed multi edge graph, where each vertex corresponds to a subject or an object.

We represent a given \href{http://www.w3.org/TR/sparql11-overview/}{SPARQL} query by a query graph Q. Query processing involves finding subgraph matches of Q over the RDF graph G, instead of joining tables in relational data management system. gStore incorporates an index over the RDF graph (called VS-tree) to speed up query processing. VS-tree is a height balanced tree with a number of associated pruning techniques to speed up subgraph matching.

\textbf{The gStore project is supported by the National Science Foundation of China (NSFC), Natural Sciences and Engineering Research Council (NSERC) of Canada, and Hong Kong RGC.}

\hyperdef{}{why-gstore}{\subsubsection{Why gStore}\label{why-gstore}}

After a series of test, we analyse and keep the result in \hyperref[chapter15]{Test Results}. gStore runs faster to answer complicated queries(for example, contain circles) than other database systems. For simple queries, both gStore and other database systems work
well.

In addition, now is the big data era and more and more structured data is coming, while the original relational database systems(or database systems based on relational tables) cannot deal with them efficiently.  In contrast, gStore can utilize the features of graph data structures, and improve the performance.

What is more, gStore is a high-extensible project. Many new ideas of graph database have be proposed, and most of them can be used in gStore.  For example, our group is also designing a distributed gstore system, which is expected to be released at the end of 2016.

\hyperdef{}{open-source}{\subsubsection{Open Source}\label{open-source}}

The gStore source code is available as open-source code under the BSD license. You are welcome to use gStore, report bugs or suggestions, or join us to make gStore better. It is also allowed for you to build all kinds of applications based on gStore, while respecting our work.

\clearpage

\hyperdef{}{chapter03}{\subsection{Chapter 03: Install Guide}\label{chapter03}}

You are advised to read init.conf file, and modify it as you wish. (this file will configure the basic options of gStore system)

gStore is a green software, and you just need to compile it with one command. Please run \texttt{make} in the gStore root directory to compile the gStore code, link the ANTLR lib, and build executable ``gbuild'', ``gquery'', ``gserver'', ``gclient'', ``gconsole''. What is more, the api of gStore is also built now.

If you want to use API examples of gStore, please run \texttt{make\ APIexample} to compile example codes for both C++ API and Java API. For details of API, please visit \hyperref[chapter05]{API} chapter.

Use \texttt{make\ clean} command to clean all objects, executables, and use \texttt{make\ dist} command to clean all objects, executables, libs, datasets, databases, debug logs, temp/text files in the gStore root directory.

You are free to modify the source code of gStore and create your own project while respecting our work, and type \texttt{make\ tarball} command to compress all useful files into a .tar.gz file, which is easy to carry.

Type \texttt{make\ gtest} to compile the gtest program if you want to use this test utility. You can see the \hyperref[chapter04]{HOW TO USE} for details of gtest program.

\clearpage

\hyperdef{}{chapter04}{\subsection{Chapter 04: How To Use}\label{chapter04}}

\textit{gStore currently includes five executables and others.}

\textbf{All the commands of gStore should be used in the root directory of gStore like bin/gconsole, because executables are placed in bin/, and they may use some files whose paths are indicated in the code, not absolute paths. We will ensure that all paths are absolute later by asking users to give the absolute path in their own systems to really install/configure the gStore. However, you must do as we told now to avoid errors.}

\hyperdef{}{0-gconsole}{\paragraph{0. gconsole}\label{0-gconsole}}

gconsole is the main console of gStore, which integrates with all functions to operate on gStore, as well as some system commands.  Completion of commands name, line editing features and access to the history list are all provided. Feel free to try it, and you may have a wonderful tour!(spaces or tabs at the beginning or end is ok, and no need to type any special characters as separators)

\begin{verbatim}
[bookug@localhost gStore]$ bin/gconsole
Gstore Console(gconsole), an interactive shell based utility to communicate with
gStore repositories.
usage: start-gconsole [OPTION]
-h,--help              print this help
-s,--source            source the SPARQL script
For bug reports and suggestions, see https://github.com/Caesar11/gStore

notice that commands are a little different between native mode and remote mode!
now is in native mode, please type your commands.
please do not use any separators in the end.

gstore>help

gstore>help drop
drop            Drop a database according to the given path.

gstore>connect 127.0.0.1 3305
now is in remote mode, please type your commands.

server>disconnect
now is in native mode, please type your commands.

gstore>build lubm_10 ./data/LUBM_10.n3
...
import RDF file to database done.

gstore>unload

gstore>load lubm_10
...
database loaded successfully!

gstore>show
lubm_10

gstore>query ./data/LUBM_q0.sql
...
final result is :
?x
<http://www.Department0.University0.edu/FullProfessor0>
<http://www.Department1.University0.edu/FullProfessor0>
<http://www.Department2.University0.edu/FullProfessor0>
<http://www.Department3.University0.edu/FullProfessor0>
<http://www.Department4.University0.edu/FullProfessor0>
<http://www.Department5.University0.edu/FullProfessor0>
<http://www.Department6.University0.edu/FullProfessor0>
<http://www.Department7.University0.edu/FullProfessor0>
<http://www.Department8.University0.edu/FullProfessor0>
<http://www.Department9.University0.edu/FullProfessor0>
<http://www.Department10.University0.edu/FullProfessor0>
<http://www.Department11.University0.edu/FullProfessor0>
<http://www.Department12.University0.edu/FullProfessor0>
<http://www.Department13.University0.edu/FullProfessor0>
<http://www.Department14.University0.edu/FullProfessor0>

gstore>query "select distinct ?x ?y where { ?x <rdf:type>
<ub:UndergraduateStudent> .
?x <ub:takesCourse> ?y . ?y <ub:name> <FullProfessor1> . }"
final result is :
?x      ?y
[empty result]

gstore>unload

gstore>quit
\end{verbatim}

Just type \texttt{bin/gconsole} in the root directory of gStore to use this console, and you will find a \texttt{gstore\textgreater{}} prompt, which indicates that you are in native mode and can type in native commands now. There are another mode of this console, which is called remote mode. Just type \texttt{connect} in the native mode to enter the remote mode, and type \texttt{disconnect} to exit to native mode.(the console connect to a gStore server whose ip is `127.0.0.1' and port is 3305, you can specify them by type \texttt{connect\ gStore\_server\_ip\ gStore\_server\_port})

You can use \texttt{help} or \texttt{?} either in native mode or remote mode to see the help information, or you can type \texttt{help\ command\_name} or \texttt{?\ command\_name} to see the information of a given command. Notice that there are some differences between the commands in native mode and commands in remote mode. For example, system commands like \texttt{ls}, \texttt{cd} and \texttt{pwd} are provided in native mode, but not in remote mode. Also take care that not all commands contained in the help page are totally achieved, and we may change some functions of the console in the future.

What we have done is enough to bring you much convenience to use gStore, just enjoy it!

\hyperdef{}{1-gbuild}{\paragraph{1. gbuild}\label{1-gbuild}}

gbuild is used to build a new database from a RDF triple format file.

\texttt{bin/gbuild\ db\_name\ rdf\_triple\_file\_path}

For example, we build a database from LUBM\_10.n3 which can be found in
example folder.

\begin{verbatim}
[bookug@localhost gStore]$ bin/gbuild LUBM10 ./data/LUBM_10.n3
gbuild...
argc: 3 DB_store:LUBM10      RDF_data: ./data/LUBM_10.n3
begin encode RDF from : ./data/LUBM_10.n3 ...
\end{verbatim}

\hyperdef{}{2-gquery}{\paragraph{2. gquery}\label{2-gquery}}

gquery is used to query an existing database with files containing
SPARQL queries.(each file contains exact one SPARQL query)

Type \texttt{bin/gquery\ db\_name\ query\_file} to execute the SPARQL
query retrieved from query\_file in the database named db\_name.

Use \texttt{bin/gquery\ -\/-help} for detail information of gquery
usage.

To enter the gquery console, type \texttt{bin/gquery\ db\_name}. The
program shows a command prompt(``gsql\textgreater{}''), and you can type
in a command here. Use \texttt{help} to see basic information of all
commands, while \texttt{help\ command\_t} shows details of a specified
command.

Type \texttt{quit} to leave the gquery console.

For \texttt{sparql} command, input a file path which contains a single
SPARQL query. (\emph{answer redirecting to file is supported})

When the program finish answering the query, it shows the command prompt
again.

\emph{gStore2.0 only support simple ``select'' queries(not for
predicates) now.}

We also take LUBM\_10.n3 as an example.

\begin{verbatim}
[bookug@localhost gStore]$ bin/gquery LUBM10
gquery...
argc: 2 DB_store:LUBM10/
loadTree...
LRUCache initial...
LRUCache initial finish
finish loadCache
finish loadEntityID2FileLineMap
open KVstore
finish load
finish loading
Type `help` for information of all commands
Type `help command_t` for detail of command_t
gsql>sparql ./data/LUBM_q0.sql
... ...
Total time used: 4ms.
final result is :
<http://www.Department0.University0.edu/FullProfessor0>
<http://www.Department1.University0.edu/FullProfessor0>
<http://www.Department2.University0.edu/FullProfessor0>
<http://www.Department3.University0.edu/FullProfessor0>
<http://www.Department4.University0.edu/FullProfessor0>
<http://www.Department5.University0.edu/FullProfessor0>
<http://www.Department6.University0.edu/FullProfessor0>
<http://www.Department7.University0.edu/FullProfessor0>
<http://www.Department8.University0.edu/FullProfessor0>
<http://www.Department9.University0.edu/FullProfessor0>
<http://www.Department10.University0.edu/FullProfessor0>
<http://www.Department11.University0.edu/FullProfessor0>
<http://www.Department12.University0.edu/FullProfessor0>
<http://www.Department13.University0.edu/FullProfessor0>
<http://www.Department14.University0.edu/FullProfessor0>
\end{verbatim}

Notice:

\begin{itemize}
\item
  ``{[}empty result{]}'' will be printed if no answer, and there is an
  empty line after all results.
\item
  readline lib is used, so you can use arrow key in your keyboard to see
  command history, and use and arrow key to move and modify your entire
  command.
\item
  path completion is supported for utility. (not built-in command
  completion)
\end{itemize}

\hyperdef{}{3-ghttp}{\paragraph{3. ghttp}\label{3-ghttp}}

ghttp is a daemon. It should be launched first when accessing gStore by HTTP protocol. It uses port 9000.

Just type \texttt{bin/ghttp} to start server. After the server is started, you can access it by visit the url in a browser or use the Restful API in your program. You can press Ctrl-C to stop the server. (Multiple connections are supported in HTTP server)

\begin{verbatim}
[bookug@localhost gStore]$ bin/ghttp
the current settings are as below:
key : value
-----------------------------------------------------------
BackupTime : 2000          # 4 am (GMT+8)
buffer_maxium : 100
db_home : .
db_suffix : .db
debug_level : simple
gstore_mode : single
operation_logs : true
thread_maxium : 1000

enter initialize.
server port: 9000 database name: 

\end{verbatim}

URL rules are listed blow:  

parameters: operation, db\_name, ds\_path, format, sparql

NOTICE: do URL encoding before sending it to database server.

operation: build, load, unload, query, monitor, show, checkpoint
\begin{itemize}
	\item 
	db\_name: the name of database, like lubm
	\item 
	format: html, json, txt, csv
	\item 
	sparql: select ?s where { ?s ?p ?o . }
	\item 
	ds\_path in the server: like /home/data/test.n3				
\end{itemize}

Examples:

\begin{itemize}
	\item 
	to build a database from a dataset:\\
	http://localhost:9000/?operation=build\&db\_name=[db\_name]\&ds\_path=[ds\_path]
	\item 
	to load a database:\\
	http://localhost:9000/?operation=load\&db\_name=[db\_name]
	\item 
	to query a database:\\
	http://localhost:9000/?operation=query\&format=[format]\&sparql=[sparql]
	\item 
	to unload a database:\\
	http://localhost:9000/?operation=unload\&db\_name=[db\_name]
	\item 
	to monitor the server:\\
	http://localhost:9000/?operation=monitor
	\item 
	to show the database used:\\
	http://localhost:9000/?operation=show
	\item 
	to save the database currently:\\
	http://localhost:9000/?operation=checkpoint		
\end{itemize}


\hyperdef{}{4-gserver}{\paragraph{4. gserver}\label{4-gserver}}

gserver is a daemon. It should be launched first when accessing gStore
by gclient or API. It communicates with client through socket.

\begin{verbatim}
[bookug@localhost gStore]$ bin/gserver -s
Server started at port 3305
\end{verbatim}

\begin{verbatim}
[bookug@localhost gStore]$ bin/gserver -t
Server stopped at port 3305
\end{verbatim}

You can also assign a custom port for listening.

\begin{verbatim}
[bookug@localhost gStore]$ bin/gserver -p 3307
Port changed to 3307.
\end{verbatim}

Notice: Multiple threads are not supported by gserver. If you start up
gclient in more than one terminal in the same time, gserver will go
down.

\hyperdef{}{5-gclient}{\paragraph{5. gclient}\label{5-gclient}}

gclient is designed as a client to send commands and receive feedbacks.

\begin{verbatim}
[bookug@localhost gStore]$ bin/gclient
ip=127.0.0.1 port=3305
gsql>help
help - print commands message
quit - quit the console normally
import - build a database for a given dataset
load - load an existen database
unload - unload an existen database
sparql - load query from the second argument
show - show the current database's name
gsql>import lubm data/LUBM_10.n3
import RDF file to database done.
gsql>load lubm
load database done.
gsql>sparql "select ?s ?o where { ?s <rdf:type> ?o . }"
[empty result]

gsql>quit
\end{verbatim}

You can also assign gserver's ip and port.

\begin{verbatim}
[bookug@localhost gStore]$ bin/gclient 172.31.19.15 3307
ip=172.31.19.15 port=3307
gsql>
\end{verbatim}

We can use these following commands now:

\begin{itemize}
\item
  \texttt{help} shows the information of all commands
\item
  \texttt{import\ db\_name\ rdf\_triple\_file\_name} build a database
  from RDF triple file
\item
  \texttt{load\ db\_name} load an existing database
\item
  \texttt{unload\ db\_name} unload database, but will not delete it on
  disk, you can load it next time
\item
  \texttt{sparql\ "query\_string"} query the current database with a
  SPARQL query string(quoted by ``'')
\item
  \texttt{show} displays the name of the current loaded database
\end{itemize}

Notice:

\begin{itemize}
\item
  at most one database can be loaded in the gclient console
\item
  you can place ` ' or `\textbackslash{}t' between different parts of
  command, but not use characters like `;'
\item
  you should not place any space or tab ahead of the start of any
  command
\end{itemize}

\hyperdef{}{6-test-utilities}{\paragraph{6. test
utilities}\label{6-test-utilities}}

A series of test program are placed in the test/ folder, and we will
introduce the two useful ones: gtest.cpp and full\_test.sh

\textbf{gtest is used to test gStore with multiple datasets and
queries.}

To use gtest utility, please type \texttt{make\ gtest} to compile the
gtest program first. Program gtest is a test tool to generate structural
logs for datasets. Please type \texttt{./gtest\ -\/-help} in the working
directory for details.

\textbf{Please change paths in the test/gtest.cpp if needed.}

You should place the datasets and queries in this way:

\begin{verbatim}
DIR/WatDiv/database/*.nt

DIR/WatDiv/query/*.sql
\end{verbatim}

Notice that DIR is the root directory where you place all datasets
waiting to be used by gtest. And WatDiv is a class of datasets, as well
as LUBM. Inside WatDiv(or LUBM, etc. please place all datasets(named
with .nt) in a database/ folder, and place all queries(corresponding to
datasets, named with .sql) in a query folder.

Then you can run the gtest program with specified parameters, and the
output will be sorted into three logs in gStore root directory:
load.log/(for database loading time and size), time.log/(for query time)
and result.log/(for all query results, not the entire output strings,
but the information to record the selected two database systems matched
or not).

All logs produced by this program are in TSV format(separated with
`\textbackslash{}t'), you can load them into Calc/Excel/Gnumeric
directly. Notice that time unit is ms, and space unit is kb.

\textbf{full\_test.sh is used to compare the performance of gStore and
other database systems on multiple datasets and queries.}

To use full\_test.sh utility, please download the database system which
you want to tats and compare, and set the exact position of database
systems and datasets in this script. The name strategy should be the
same as the requirements of gtest, as well as the logs strategy.

Only gStore and Jena are tested and compared in this script, but it is
easy to add other database systems, if you would like to spend some time
on reading this script. You may go to
\href{run:../pdf/gstore���Ա���.pdf}{test
report} or \hyperref[chapter11]{Frequently Asked Questions} for help if
you encounter a problem.

\hyperdef{}{7-gadd}{\paragraph{7. gadd}\label{7-gadd}}

gadd is used to add triples in a file to an existing database.

Usage: \texttt{bin/gadd db\_name rdf\_triple\_file\_path}.

\begin{verbatim}
[bookug@localhost gStore]$ bin/gadd lubm ./data/LUBM\_10.n3
...
argc: 3 DB_store:lubm   insert file:./data/LUBM_10.n3
get important pre ID
...
insert rdf triples done.
inserted triples num: 99550
\end{verbatim}

\hyperdef{}{8-gsub}{\paragraph{8. gsub}\label{8-gsub}}

gsub is used to remove triples from an existing database.

Usage: \texttt{bin/gsub db\_name rdf\_triple\_file\_path}.

\begin{verbatim}
[bookug@localhost gStore]$ bin/gsub lubm data/LUBM\_10.n3
...
argc: 3 DB_store:lubm  remove file: data/LUBM\_10.n3
...
remove rdf triples done.
removed triples num: 99550
\end{verbatim}

\hyperdef{}{9-gmonitor}{\paragraph{9. gmonitor}\label{9-gmonitor}}

After starting ghttp, go into gStore/bin/ and type \texttt{./gmonitor ip port} to check current status of gStore.

\begin{verbatim}
[bookug@localhost bin]$ ./gmonitor 127.0.0.1 9000
parameter: ?operation=monitor
request: http://127.0.0.1:9000/%3Foperation%3Dmonitor
null--->[HTTP/1.1 200 OK]
Content-Length--->[127]
database: lubm
triple num: 99550
entity num: 28413
literal num: 0
subject num: 14569
predicate num: 17
connection num: 7
\end{verbatim}

\hyperdef{}{10-gshow}{\paragraph{10. gshow}\label{10-gshow}}

After starting ghttp, go into gStore/bin and type \texttt{./gshow ip port} to check loaded database.

\begin{verbatim}
[bookug@localhost gStore]$ ./gshow 127.0.0.1 9000
parameter: ?operation=show
request: http://127.0.0.1:9000/%3Foperation%3Dshow
null--->[HTTP/1.1 200 OK]
Content-Length--->[4]
lubm
\end{verbatim}

\clearpage

\part{Advanced}

\hyperdef{}{chapter05}{\subsection{Chapter 05: Socket API Explanation}\label{chapter05}}

\textbf{This Chapter guides you to use socket API for accessing gStore, which can be used when the server runs gserver. We also provide HTTP API for ghttp, please see \hyperref[chapter06]{【HTTP API Explanation】}.}

\hyperdef{}{easy-examples}{\subsubsection{Easy
Examples}\label{easy-examples}}

We provide JAVA, C++, PHP and Python API for gStore now. Please refer to example
codes in \texttt{api/socket/cpp/example}, \texttt{api/socket/java/example}, \texttt{api/socket/php} and \texttt{api/socket/python/example}. To use the four examples to have a try, please ensure that executables have already been generated. Otherwise, for Java and C++, just type \texttt{make\ APIexample} in the root directory of gStore to compile the codes, as well as API.

Next, \textbf{start up a gStore server by using \texttt{./gserver}
command.} It is ok if you know a running usable gStore server and try to
connect to it, but notice that \textbf{the server ip and port of server
and client must be matched.}(you don't need to change any thing if using
examples, just by default) Then, for Java and C++ code, you need to compile the example codes
in the directory gStore/api/socket/. We provide a utility to do this, and you
just need to type \texttt{make\ APIexample} in the root directory of
gStore. Or you can compile the codes by yourself, in this case please go
to gStore/api/socket/cpp/example/ and gStore/api/socket/java/example/, respectively.

Finally, go to the example directory and run the corresponding
executables. For C++, just use \texttt{./example} command to run it. And
for Java, use \texttt{make\ run} command or \texttt{java\ -cp\ ../lib/GstoreJavaAPI.jar:.\ JavaAPIExample} to run
it. For PHP, use \texttt{php ./PHPAPIExample}. For python, use \texttt{python ./PythonAPIExample}. All these four executables will connect to a specified gStore server
and do some load or query operations. Be sure that you see the query
results in the terminal where you run the examples, otherwise please go
to \hyperref[chapter11]{Frequently Asked Questions} for help or report
it to us.(the report approach is described in
\hyperref[chapter00]{README})

You are advised to read the example code carefully, as well as the
corresponding Makefile. This will help you to understand the API,
specially if you want to write your own programs based on the API
interface.

\hyperdef{}{api-structure}{\subsubsection{API structure}\label{api-structure}}

The API of gStore is placed in api/socket/ directory in the root directory of
gStore, whose contents are listed below:

\begin{itemize}
\item
  gStore/api/socket/

  \begin{itemize}
  \item
    cpp/ (the C++ API)

    \begin{itemize}
    \item
      src/ (source code of C++ API, used to build the
      lib/libgstoreconnector.a)

      \begin{itemize}
      \item
        GstoreConnector.cpp (interfaces to interact with gStore server)
      \item
        GstoreConnector.h
      \item
        Makefile (compile and build lib)
      \end{itemize}
    \item
      lib/ (where the static lib lies in)

      \begin{itemize}
      \item
        .gitignore
      \item
        libgstoreconnector.a (only exist after compiled, you need to
        link this lib when you use the C++ API)
      \end{itemize}
    \item
      example/ (small example program to show the basic idea of using
      the C++ API)

      \begin{itemize}
      \item
        CppAPIExample.cpp
      \item
        Makefile
      \end{itemize}
    \end{itemize}
  \item
    java/ (the Java API)

    \begin{itemize}
    \item
      src/ (source code of Java API, used to build the
      lib/GstoreJavaAPI.jar)

      \begin{itemize}
      \item
        jgsc/GstoreConnector.java (the package which you need to import when you use the Java API)
      \item
        Makefile (compile and build lib)
      \end{itemize}
    \item
      lib/

      \begin{itemize}
      \item
        .gitignore
      \item
        GstoreJavaAPI.jar (only exist after compiled, you need to
        include this JAR in your class path)
      \end{itemize}
    \item
      example/ (small example program to show the basic idea of using
      the Java API)

      \begin{itemize}
      \item
        JavaAPIExample.cpp
      \item
        Makefile
      \end{itemize}
    \end{itemize}

   \item
    php/ (the PHP API)

    \begin{itemize}
    \item
      GstoreConnector.php (source code of PHP API, you need to include this file when you use the PHP API)

    \item
      PHPAPIExample.php (small example program to show the basic idea of using the PHP API)
    \end{itemize}

  \item
    python/ (the Python API)
    \begin{itemize}
    \item
      src/ (source code of Python API)
      \begin{itemize}
      \item
        GstoreConnector.py (the package which you need to import when you use the Python API)
      \end{itemize}
    \item
      example/ (small example program to show the basic idea of using the Python API)
      \begin{itemize}
      \item
        PythonAPIExample.py
      \end{itemize}
    \end{itemize}

  \end{itemize}
\end{itemize}

\hyperdef{}{c-api}{\subsubsection{C++ API}\label{c-api}}

\hyperdef{}{interface}{\paragraph{Interface}\label{interface}}

To use the C++ API, please place the phrase
\texttt{\#include\ "GstoreConnector.h"} in your cpp code. Functions in
GstoreConnector.h should be called like below:

\begin{verbatim}
// initialize the Gstore server's IP address and port.
GstoreConnector gc("127.0.0.1", 3305);
// build a new database by a RDF file.
// note that the relative path is related to gserver.
gc.build("LUBM10", "example/LUBM_10.n3");
// then you can execute SPARQL query on this database.
std::string sparql = "select ?x where \
{\
?x    <rdf:type>    <ub:UndergraduateStudent>. \
?y    <ub:name> <Course1>. \
?x    <ub:takesCourse>  ?y. \
?z    <ub:teacherOf>    ?y. \
?z    <ub:name> <FullProfessor1>. \
?z    <ub:worksFor>    ?w. \
?w    <ub:name>    <Department0>. \
}";
std::string answer = gc.query(sparql);
// unload this database.
gc.unload("LUBM10");
// also, you can load some exist database directly and then query.
gc.load("LUBM10");
// query a SPARQL in current database
answer = gc.query(sparql);
\end{verbatim}

The original declaration of these functions are as below:

\begin{verbatim}
GstoreConnector();
GstoreConnector(string _ip, unsigned short _port);
GstoreConnector(unsigned short _port);
bool load(string _db_name);
bool unload(string _db_name);
bool build(string _db_name, string _rdf_file_path);
string query(string _sparql);
\end{verbatim}

Notice:

\begin{enumerate}
\item
  When using GstoreConnector(), the default value for ip and port is
  127.0.0.1 and 3305, respectively.
\item
  When using build(), the rdf\_file\_path(the second parameter) should
  be related to the position where gserver lies in.
\item
  Please remember to unload the database you have loaded, otherwise
  things may go wrong.(the errors may not be reported!)
\end{enumerate}

\hyperdef{}{compile}{\paragraph{Compile}\label{compile}}

You are advised to see gStore/api/socket/cpp/example/Makefile for instructions on how to compile your code with the C++ API. Generally, what you must do is compile your own code to object with header in the C++ API, and link the object with static lib in the C++ API.

Let us assume that your source code is placed in test.cpp, whose position is \$\{GSTORE\}/gStore/.(if using devGstore as name instead of gStore, then the path is \$\{GSTORE\}/devGstore/ directory first:

\begin{quote}
Use \texttt{g++\ -c\ -I\$\{GSTORE\}/gStore/api/socket/cpp/src/\ test.cpp\ -o\ test.o} to compile your test.cpp into test.o, relative API header is placed in api/socket/cpp/src/.

Use \texttt{g++\ -o\ test\ test.o\ -L\$\{GSTORE\}/gStore/api/socket/cpp/lib/\ -lgstoreconnector} to link your test.o with the libgstoreconnector.a(a static lib) in api/socket/cpp/lib/.
\end{quote}

Then you can type \texttt{./test} to execute your own program, which uses our C++ API. It is also advised for you to place relative compile commands in a Makefile, as well as other commands if you like.

\hyperdef{}{java-api}{\subsubsection{Java API}\label{java-api}}

\hyperdef{}{interface-1}{\paragraph{Interface}\label{interface-1}}

To use the Java API, please place the phrase
\texttt{import\ jgsc.GstoreConnector;} in your java code. Functions in
GstoreConnector.java should be called like below:

\begin{verbatim}
// initialize the Gstore server's IP address and port.
GstoreConnector gc = new GstoreConnector("127.0.0.1", 3305);
// build a new database by a RDF file.
// note that the relative path is related to gserver.
gc.build("LUBM10", "example/LUBM_10.n3");
// then you can execute SPARQL query on this database.
String sparql = "select ?x where " + "{" +
"?x    <rdf:type>    <ub:UndergraduateStudent>. " +
"?y    <ub:name> <Course1>. " +
"?x    <ub:takesCourse>  ?y. " +
"?z    <ub:teacherOf>    ?y. " +
"?z    <ub:name> <FullProfessor1>. " +
"?z    <ub:worksFor>    ?w. " +
"?w    <ub:name>    <Department0>. " +
"}";
String answer = gc.query(sparql);
//unload this database.
gc.unload("LUBM10");
//also, you can load some exist database directly and then query.
gc.load("LUBM10");// query a SPARQL in current database
answer = gc.query(sparql);
\end{verbatim}

The original declaration of these functions are as below:

\begin{verbatim}
GstoreConnector();
GstoreConnector(string _ip, unsigned short _port);
GstoreConnector(unsigned short _port);
bool load(string _db_name);
bool unload(string _db_name);
bool build(string _db_name, string _rdf_file_path);
string query(string _sparql);
\end{verbatim}

Notice:

\begin{enumerate}
\item
  When using GstoreConnector(), the default value for ip and port is
  127.0.0.1 and 3305, respectively.
\item
  When using build(), the rdf\_file\_path(the second parameter) should
  be related to the position where gserver lies in.
\item
  Please remember to unload the database you have loaded, otherwise
  things may go wrong.(the errors may not be reported!)
\end{enumerate}

\hyperdef{}{compile-1}{\paragraph{Compile}\label{compile-1}}

You are advised to see gStore/api/socket/java/example/Makefile for instructions on how to compile your code with the Java API. Generally, what you must do is compile your own code to object with jar file in the Java API.

Let us assume that your source code is placed in test.java, whose position is \$\{GSTORE\}/gStore/.(if using devGstore as name instead of gStore, then the path is \$\{GSTORE\}/devGstore/ directory first:

\begin{quote}
Use \texttt{javac\ -cp\ \$\{GSTORE\}/gStore/api/socket/java/lib/GstoreJavaAPI.jar\ test.java} to compile your test.java into test.class with the GstoreJavaAPI.jar(a jar package used in Java) in api/socket/java/lib/.
\end{quote}

Then you can type \texttt{java\ -cp\ \$\{GSTORE\}/gStore/api/socket/java/lib/GstoreJavaAPI.jar:.\ test} to execute your own program(notice that the ``:.'' in command cannot be neglected), which uses our Java API. It is also advised for you to place relative compile commands in a Makefile, as well as other commands if you like.

\hyperdef{}{php-api}{\subsubsection{PHP API}\label{php-api}}
	
	\hyperdef{}{interface-2}{\paragraph{Interface}\label{interface-2}}
	
	To use the PHP API, please place the phrase
\texttt{include('GstoreConnector,php');} in your php code. Functions in
GstoreConnector.php should be called like below:
	
	\begin{verbatim}
	// initialize the Gstore server's IP address and port.
	$gc = new Connector("127.0.0.1", 3305);
	// build a new database by a RDF file.
	// note that the relative path is related to gserver.
	$gc->build("LUBM10", "example/LUBM_10.n3");
	// then you can execute SPARQL query on this database.
	$sparql = "select ?x where " + "{" +
	"?x    <rdf:type>    <ub:UndergraduateStudent>. " +
	"?y    <ub:name> <Course1>. " +
	"?x    <ub:takesCourse>  ?y. " +
	"?z    <ub:teacherOf>    ?y. " +
	"?z    <ub:name> <FullProfessor1>. " +
	"?z    <ub:worksFor>    ?w. " +
	"?w    <ub:name>    <Department0>. " +
	"}";
	$answer = gc->query($sparql);
	//unload this database.
	$gc->unload("LUBM10");
	//also, you can load some exist database directly and then query.
	$gc->load("LUBM10");// query a SPARQL in current database
	$answer = gc->query(sparql);
	\end{verbatim}
	
	The original declaration of these functions are as below:
	
	\begin{verbatim}
	class Connector {
		public function __construct($host, $port);
		public function send($data);
		public function recv();
		public function build($db_name, $rdf_file_path);
		public function load($db_name);
		public function unload($db_name);
		public function query($sparql);
		public function __destruct();
	}
	\end{verbatim}
	
	Notice:
	
	\begin{enumerate}
		\item
		When using Connector(), the default value for ip and port is
		127.0.0.1 and 3305, respectively.
		\item
		When using build(), the rdf\_file\_path(the second parameter) should
		be related to the position where gserver lies in.
		\item
		Please remember to unload the database you have loaded, otherwise
		things may go wrong.(the errors may not be reported!)
	\end{enumerate}
	
	\hyperdef{}{run-2}{\paragraph{Run}\label{run-2}}
	
	You can see gStore/api/socket/php/PHPAPIExample for instructions on how to use PHP API. PHP script doesn't need compiling. You can run PHP file directly or use it in your web project.
	

\hyperdef{}{python-api}{\subsubsection{Python API}\label{python-api}}
	
	\hyperdef{}{interface-3}{\paragraph{Interface}\label{interface-3}}
	
	To use the Python API, please place the phrase \texttt{from GstoreConnector import GstoreConnector} in your python code. Functions in GstoreConnector.py should be called like below:
	
	\begin{verbatim}
	// initialize the Gstore server's IP address and port.
	gc = GstoreConnector('127.0.0.1', 3305)
	// build a new database by a RDF file.
	// note that the relative path is related to gserver.
	gc.build('LUBM10', 'data/LUBM_10.n3')
	// then you can execute SPARQL query on this database.
	$sparql = "select ?x where " + "{" +
	"?x    <rdf:type>    <ub:UndergraduateStudent>. " +
	"?y    <ub:name> <Course1>. " +
	"?x    <ub:takesCourse>  ?y. " +
	"?z    <ub:teacherOf>    ?y. " +
	"?z    <ub:name> <FullProfessor1>. " +

	"?z    <ub:worksFor>    ?w. " +
	"?w    <ub:name>    <Department0>. " +
	"}";
	answer = gc.query(sparql)
	//unload this database.
	gc.unload('LUBM10')
	//also, you can load some exist database directly and then query.
	gc.load('LUBM10')// query a SPARQL in current database
	answer = gc.query(sparql)
	\end{verbatim}
	
	The original declaration of these functions are as below:
	
	\begin{verbatim}
	class GstoreConnector {
		def _connect(self)
		def _disconnect(self)
		def _send(self, msg):
		def _recv(self)
		def _pack(self, msg):
		def _communicate(f):
		def __init__(self, ip='127.0.0.1', port=3305):
		@_communicate
    		def test(self)
		@_communicate
		def load(self, db_name)
		@_communicate
		def unload(self, db_name)
		@_communicate
		def build(self, db_name, rdf_file_path)
		@_communicate
		def drop(self, db_name)
		@_communicate
		def stop(self)
		@_communicate
		def query(self, sparql)
		@_communicate
		def show(self, _type=False)
	}
	\end{verbatim}
	
	Notice:
	
	\begin{enumerate}
		\item
		When using GstoreConnector(), the default value for ip and port is
		127.0.0.1 and 3305, respectively.
		\item
		When using build(), the rdf\_file\_path(the second parameter) should
		be related to the position where gserver lies in.
		\item
		Please remember to unload the database you have loaded, otherwise
		things may go wrong.(the errors may not be reported!)
	\end{enumerate}
	
	\hyperdef{}{run-3}{\paragraph{Run}\label{run-3}}
	
	You are advised to see gStore/api/socket/python/example/PythonAPIExample for examples on how to use python API. Python file doesn't need compiling, and you can run it directly.

\clearpage


\hyperdef{}{chapter06}{\subsection{Chapter 06: HTTP API Explanation}\label{chapter06}}

\textbf{This chapter provides API for ghttp. Compared with socket API, HTTP API is more stable and more standard, and can maintain connection. Socket API can not guaratee correct transmission, so the network transmission is faster.}

\hyperdef{}{easy-http-examples}{\subsubsection{Easy Examples}\label{easy-http-examples}}

We provide JAVA and C++ API for ghttp now. Please see \texttt{api/http/cpp} and \texttt{api/http/java}. To use these examples, please make sure that executables have already been generated.

Next, \textbf{start up ghttp service by using \texttt{./ghttp} command.} It is ok if you know a running usable ghttp server and try to connect to it. (you don't need to change anything if using
examples, just by default) Then, for Java and C++ code, you need to compile the example codes in the directory gStore/api/http/. We provide a utility to do this, and you just need to type \texttt{make\ APIexample} in the root directory of gStore. Or you can compile the codes by yourself, in this case please go to gStore/api/http/cpp/ and gStore/api/http/java/, respectively.

Finally, go to the example directory and run the corresponding executables. All these four executables will connect to a specified ghttp server and do some load or query operations. Be sure that you see the query results in the terminal where you run the examples, otherwise please go to \hyperref[chapter11]{Frequently Asked Questions} for help or report it to us.(the report approach is described in \hyperref[chapter00]{README})

You are advised to read the example code carefully, as well as the corresponding Makefile. This will help you to understand the API, specially if you want to write your own programs based on the API interface.

\hyperdef{}{http-api-structure}{\subsubsection{API Structure}\label{http-api-structure}}

The HTTP API of gStore is placed in api/http/ directory in the root directory of gStore, whose contents are listed below:

\begin{itemize}
	\item
	gStore/api/http/
	
	\begin{itemize}
		\item
		cpp/ (C++ API)
		
		\begin{itemize}
			\item
			client.cpp (source code of C++ API)
			\item 
			client.h
			\item 
			example.cpp (example program to show the basic idea of using the C++ API)
			\item 
			Makefile (compile)
			
		\end{itemize}
		
		\item
		java/ (Java API)
		\begin{itemize}
			\item
			src/ (source code of Java API, used to build the
			lib/GstoreJavaAPI.jar)
			
			\begin{itemize}
				\item
				jgsc/GstoreConnector.java (the package which you need to import when you use the Java API)
				\item
				Makefile (compile and build lib)
			\end{itemize}
			\item
			lib/
			
			\begin{itemize}
				\item
				.gitignore
				\item
				GstoreJavaAPI.jar (only exist after compiled, you need to
				include this JAR in your class path)
			\end{itemize}
			\item
			example/ (small example program to show the basic idea of using
			the Java API)
			
			\begin{itemize}
				\item
				JavaAPIExample.cpp
				\item
				Makefile
			\end{itemize}
		\end{itemize}
		
	\end{itemize}
\end{itemize}

\hyperdef{}{http-c-api}{\subsubsection{C++ API}\label{http-c-api}}

\hyperdef{}{http-interface}{\paragraph{Interface}\label{http-interface}}


To use the C++ API, please place the phrase \texttt{\#include\ "Client.h"} in your cpp code. Functions in Client.h should be called like below:

\begin{verbatim}
CHttpClient hc;
string res; 
int ret;
// build a new database by a RDF file.
ret = hc.Get("127.0.0.1:9000/build/lumb/data/LUBM_10.n3", res);
cout<<res<<endl;
// load databse
ret = hc.Get("127.0.0.1:9000/load/lumb", res);
cout<<res<<endl;
// then you can execute SPARQL query on this database.
ret = hc.Get("127.0.0.1:9000/query/data/ex0.sql", res);
cout<<res<<endl;
// output information of current database
ret = hc.Get("127.0.0.1:9000/monitor", res);
cout<<res<<endl;
// unload this databse
ret = hc.Get("127.0.0.1:9000/unload", res);
cout<<res<<endl;
\end{verbatim}

The original declaration of these functions are as below:

\begin{verbatim}
CHttpClient();
int Post(const std::string & strUrl, const std::string & strPost, std::string & strResponse);
int Get(const std::string & strUrl, std::string & strResponse);
int Posts(const std::string & strUrl, const std::string & strPost, std::string & strResponse, const char * pCaPath = NULL);
int Gets(const std::string & strUrl, std::string & strResponse, const char * pCaPath = NULL);
\end{verbatim}


\hyperdef{}{http-java-api}{\subsubsection{Java API}\label{http-java-api}}

\hyperdef{}{http-interface-1}{\paragraph{Interface}\label{http-interface-1}}

To use the Java API, please place the phrase
\texttt{import\ jgsc.GstoreConnector;} in your java code. Functions in
GstoreConnector.java should be called like below:

\begin{verbatim}
// initialize the Gstore server's IP address and port.
GstoreConnector gc = new GstoreConnector("127.0.0.1", 9000);
// build a new database by a RDF file.
// note that the relative path is related to gserver.
gc.build("LUBM10", "example/LUBM_10.n3");
gc.load("LUBM10");
// then you can execute SPARQL query on this database.
String sparql = "select ?x where " + "{" +
"?x    <rdf:type>    <ub:UndergraduateStudent>. " +
"?y    <ub:name> <Course1>. " +
"?x    <ub:takesCourse>  ?y. " +
"?z    <ub:teacherOf>    ?y. " +
"?z    <ub:name> <FullProfessor1>. " +
"?z    <ub:worksFor>    ?w. " +
"?w    <ub:name>    <Department0>. " +
"}";
String answer = gc.query(sparql);
//unload this database.
gc.unload("LUBM10");
//also, you can load some exist database directly and then query.
gc.load("LUBM10");// query a SPARQL in current database
answer = gc.query(sparql);
gc.unload("LUBM10");
\end{verbatim}

The original declaration of these functions are as below:

\begin{verbatim}
GstoreConnector();
GstoreConnector(int _port);
GstoreConnector(String _ip, int _port);
boolean load(String _db_name);
boolean unload(String _db_name);
boolean build(String _db_name, String _rdf_file_path);
boolean drop(String _db_name);
String query(String _sparql);
String show();
String show(boolean _type);
\end{verbatim}

\clearpage


\hyperdef{}{chapter07}{\subsection{Chapter 07: Use gStore in Web}\label{chapter07}}
	
	\textbf{This Chapter provides a specific example on how to use our API in a web project.}
	
	\hyperdef{}{example}{\subsubsection{Example}\label{example}}
	
	Now you have the basic idea on how to use our APIs to connect gStore. Yet you might be still a little confused. Here we provide a simple demo to show you what to do explicitly.

	Let's say, you need to use gStore in a web project. PHP is a popular general-purpose scripting language that is especially suited to web development. So, using our PHP API can meet your requirements. Here is what we implement: http://59.108.48.18/Gstore/form.php.

	First, get your web server ready so it can run PHP files. We won't give detailed instructions on this step here. You can easily google it according to your web server(for example, Apache or Nginx, etc.) 
	
	Next, go to your web document root(usually in /var/www/html or apache/htdocs, you can check it in config file), and create a folder named "Gstore". Then copy the GstoreConnector.php file into it. Create a "PHPAPI.php" file. Edit it like below:

	\begin{verbatim}
	<?php

	include( 'GstoreConnector.php');
	$host = '127.0.0.1';
	$port = 3305;
	$dbname = $_POST["databasename"];
	$sparql = $_POST["sparql"];
	$format = $_POST["format"];
	$load = new Connector($host, $port);
	$load->load($dbname);
	$query = new Connector($host, $port);
	$result = $query->query($sparql);
	switch ($format) {
    	case 1:
        $array = explode("<", $result);
        $html = '<html><table class="sparql" border="1"><tr><th>' . 
                $array[0] . "</th></tr>";
        for ($i = 1; $i < count($array); $i++) {
            $href = str_replace(">", "", $array[$i]);
            $html.= '<tr><td><a href="' . $href . '">' . 
                    $href . '</a></td></tr>';
        }
        $html.= '</table></html>';
        echo $html;
        exit;

    	case 2:
        $filename = 'result.txt';
        header("Content-Type: application/octet-stream");
        header('Content-Disposition: attachment; 
                filename="' . $filename . '"');
        echo $result;
        exit;

    	case 3:
        $filename = 'result.csv';
        header("Content-Type: application/octet-stream");
        header('Content-Disposition: attachment; 
                filename="' . $filename . '"');
        $array = explode("<", $result);
        echo $array[0];
        for ($i = 1; $i < count($array); $i++) {
            $href = str_replace(">", "", $array[$i]);
            echo $href;
        	}
        exit;
	}
	?>
	\end{verbatim}

	This PHP file get three parametres from a website, including databasename, sparql and output format. Then it use our PHP API to connect gStore and run the query. Finally, the "switch" part gives the output. 

	After that, we need a website to collect those imformation(databasename, sparql and output format). We create a html file and use a form to do it, just like below:
	\begin{verbatim}
<form id="form_1145884" class="appnitro"  method="post" action="PHPAPI.php">
  <div class="form_description">
    <h2>Gstore SPARQL Query Editor</h2>
    <p></p>
  </div>						
  <ul>
    <li id="li_1" >
      <label class="description" for="element_1">
        Database Name 
      </label>
      <div>
        <input id="element_1" name="databasename" class="element text medium" 
         type="text" maxlength="255" value="dbpedia_2014_reduce">
        </input>
      </div> 
    </li>		

    <li id="li_3">
      <label class="description" for="element_3">Query Text </label>
      <div>
        <textarea id="element_3" name="sparql" class="element textarea large">
          SELECT DISTINCT ?uri 
          WHERE { 
            ?uri <type> <Astronaut> . 
          { ?uri <nationality> <Russia> . } 
          UNION 
          { ?uri <nationality> <Soviet_Union> . } 
          } 
      </textarea> 
      </div> 
    </li>		

    <li id="li_5" >
      <label class="description" for="element_5">
        Results Format 
      </label>
      <div>
        <select class="element select medium" id="element_5" name="format"> 
          <option value="1" selected="ture">HTML</option>
          <option value="2" >Text</option>
          <option value="3" >CSV</option>
        </select>
      </div> 

    <li class="buttons">
      <input type="hidden" name="form_id" value="1145884" />
      <input id="saveForm" class="button_text" type="submit" 
       name="submit" value="Run Query" />
    </li>
  </ul>
</form>
	\end{verbatim}
	
	As you can see in the code, we use a <input> element to get the databasename, and <texarea> for sparql, <select> for output format. <form> lable has an attribute "action" which specifies which file to execute. So, when you click the "submit" button, it will call PHPAPI.php file and post the values from the form. 

	Finally, don't forget to start gserver on your server.
	
	\clearpage


\hyperdef{}{chapter08}{\subsection{Chapter 08: Project Structure}\label{chapter08}}

\textbf{This chapter introduce the whole structure of the gStore system project.}

\hyperdef{}{the-core-source-codes}{\paragraph{The core source codes are listed below:}\label{the-core-source-codes}}

\begin{itemize}
\item
  Database/ (calling other core parts to deal with requests from
  interface part)

  \begin{itemize}
  \item
    Database.cpp (achieve functions)
  \item
    Database.h (class, members and functions definitions)
  \item
    Join.cpp (join the node candidates to get results)
  \item
    Join.h (class, members,, and functions definitions)
  \end{itemize}
\item
  KVstore/ (a key-value store to swap between memory and disk)

  \begin{itemize}
  \item
    KVstore.cpp (interact with upper layers)
  \item
    KVstore.h
  \item
    heap/ (a heap of nodes whose content are in memory)

    \begin{itemize}
    \item
      Heap.cpp
    \item
      Heap.h
    \end{itemize}
  \item
    node/ (all kinds of nodes in B+-tree)

    \begin{itemize}
    \item
      Node.cpp (the base class of IntlNode and LeafNode)
    \item
      Node.h
    \item
      IntlNode.cpp (internal nodes in B+-tree)
    \item
      IntlNode.h
    \item
      LeafNode.cpp (leaf nodes in B+-tree)
    \item
      LeafNode.h
    \end{itemize}
  \item
    storage/ (swap contents between memory and disk)

    \begin{itemize}
    \item
      file.h
    \item
      Storage.cpp
    \item
      Storage.h
    \end{itemize}
  \item
    tree/ (implement all tree operations and interfaces)

    \begin{itemize}
    \item
      Tree.cpp
    \item
      Tree.h
    \end{itemize}
  \end{itemize}
\item
  Query/ (needed to answer SPARQL query)

  \begin{itemize}
  \item
    BasicQuery.cpp (basic type of queries without aggregate operations)
  \item
    BasicQuery.h
  \item
    IDList.cpp (candidate list of a node/variable in query)
  \item
    IDList.h
  \item
    ResultSet.cpp (keep the result set corresponding to a query)
  \item
    ResultSet.h
  \item
    SPARQLquery.cpp (deal with a entire SPARQL query)
  \item
    SPARQLquery.h
  \item
    Varset.cpp
  \item
    Varset.h
  \item
    QueryTree.cpp
  \item
    QueryTree.h
  \item
    GeneralEvaluation.cpp
  \item
    GeneralEvaluation.h
  \item
    RegexExpression.h
  \end{itemize}
\item
  Signature/ (assign signatures for nodes and edges, but not for
  literals)

  \begin{itemize}
  \item
    SigEntry.cpp
  \item
    SigEntry.h
  \item
    Signature.cpp
  \item
    Signature.h
  \end{itemize}
\item
  VSTree/ (an tree index to prune more efficiently)

  \begin{itemize}
  \item
    EntryBuffer.cpp
  \item
    EntryBuffer.h
  \item
    LRUCache.cpp
  \item
    LRUCache.h
  \item
    VNode.cpp
  \item
    VNode.h
  \item
    VSTree.cpp
  \item
    VSTree.h
  \end{itemize}
\end{itemize}

\hyperdef{}{the-parser-part}{\paragraph{The parser part is listed below:}\label{the-parser-part}}

\begin{itemize}
\item
  Parser/

  \begin{itemize}
  \item
    DBParser.cpp
  \item
    DBParser.h
  \item
    RDFParser.cpp
  \item
    RDFParser.h
  \item
    SparqlParser.c (auto-generated, subtle modified manually,
    compressed)
  \item
    SparqlParser.h (auto-generated, subtle modified manually,
    compressed)
  \item
    SparqlLexer.c (auto-generated, subtle modified manually, compressed)
  \item
    SparqlLexer.h (auto-generated, subtle modified manually, compressed)
  \item
    TurtleParser.cpp
  \item
    TurtleParser.h
  \item
    Type.h
  \item
    QueryParser.cpp
  \item
    QueryParser.h
  \end{itemize}
\end{itemize}

\hyperdef{}{the-utilities}{\paragraph{The utilities are listed below:}\label{the-utilities}}

\begin{itemize}
\item
  Util/

  \begin{itemize}
  \item
    Util.cpp (headers, macros, typedefs, functions\ldots{})
  \item
    Util.h
  \item
    Bstr.cpp (represent strings of arbitrary length)
  \item
    Bstr.h (class, members and functions definitions)
  \item
    Stream.cpp (store and use temp results, which may be very large)
  \item
    Stream.h
  \item
    Triple.cpp (deal with triples, a triple can be divided as
    subject(entity), predicate(entity), object(entity or literal))
  \item
    Triple.h
  \item
    BloomFilter.cpp
  \item
    BloomFilter.h
  \end{itemize}
\end{itemize}

\hyperdef{}{the-interface-part}{\paragraph{The interface part is listed below:}\label{the-interface-part}}

\begin{itemize}
\item
  Server/ (client and server mode to use gStore)

  \begin{itemize}
  \item
    Client.cpp
  \item
    Client.h
  \item
    Operation.cpp
  \item
    Operation.h
  \item
    Server.cpp
  \item
    Server.h
  \item
    Socket.cpp
  \item
    Socket.h
  \end{itemize}
\item
  Main/ (a series of applications/main-program to operate on gStore)

  \begin{itemize}
  \item
    gbuild.cpp (import a RDF dataset)
  \item
    gquery.cpp (query a database)
  \item
    gserver.cpp (start up the gStore server)
  \item
    gclient.cpp (connect to a gStore server and interact)
  \end{itemize}
\end{itemize}

\hyperdef{}{more-details}{\paragraph{More details}\label{more-details}}

To acquire a deep understanding of gStore codes, please go to
\href{run:../pdf/code_overview.pdf}{Code
Detail}. See
\href{run:../pdf/Gstore2.0_useCaseDoc.pdf}{use
case} to understand the design of use cases, and see
\href{run:../pdf/OOA_class.pdf}{OOA}
and
\href{run:../pdf/OOD_class.pdf}{OOD}
for OOA design and OOD design, respectively.

If you want to know the sequence of a running gStore, please view the
list below:

\begin{itemize}
\item
  \href{run:../jpg/A01-connectServer.jpg}{connect
  to server}
\item
  \href{run:../jpg/A02-disconnectServer.jpg}{disconnect
  server}
\item
  \href{run:../jpg/A03-loadDatabase.jpg}{load
  database}
\item
  \href{run:../jpg/A04-unloadDatabase.jpg}{unload
  database}
\item
  \href{run:../jpg/A05-buildDatabase.jpg}{create
  database}
\item
  \href{run:../jpg/A06-deleteDatabase.jpg}{delete
  database}
\item
  \href{run:../jpg/A07-connectDatabase.jpg}{connect
  to database}
\item
  \href{run:../jpg/A08-disconnectDatabase.jpg}{disconnect
  database}
\item
  \href{run:../jpg/A09-showDatabase.jpg}{show
  databases}
\item
  \href{run:../jpg/A10-querySPARQL.jpg}{SPARQL
  query}
\item
  \href{run:../jpg/A11-loadRDF.jpg}{import
  RDF dataset}
\item
  \href{run:../jpg/A12-insertRDF.jpg}{insert
  a triple}
\item
  \href{run:../jpg/A13-deleteRDF.jpg}{delete
  a triple}
\item
  \href{run:../jpg/B01-createAccount.jpg}{create
  account}
\item
  \href{run:../jpg/B02-deleteAccount.jpg}{delete
  account}
\item
  \href{run:../jpg/B03-changeAccount.jpg}{modify
  account authority}
\item
  \href{run:../jpg/B04-removeDatabase.jpg}{compulsively
  unload database}
\item
  \href{run:../jpg/B05-showAccount.jpg}{see
  account authority}
\end{itemize}

It is really not strange to see something different with the original
design in the source code. And some designed functions may have not be
achieved so far.

\hyperdef{}{others}{\paragraph{Others}\label{others}}

The api/ folder in gStore is used to store API program, libs and
examples, please go to \hyperref[chapter05]{API} for details. And test/
is used to store a series test programs or utilities, such as gtest,
full\_test and so on. Chapters related with test/ are
\hyperref[chapter04]{How To Use} and \hyperref[chapter15]{Test Result}.
This project need an ANTLR lib to parse the SPARQL query, whose code is
placed in tools/(also archived here) and the compiled libantlr.a is
placed in lib/ directory.

We place some datasets and queries in data/ directory as examples, and
you can try them to see how gStore works. Related instructions are in
\hyperref[chapter04]{How To Use}. The docs/ directory contains all kinds
of documents of gStore, including a series of markdown files and two
folders, pdf/ and jpg/. Files whose type is pdf are placed in pdf/
folder, while files with jpg type are placed in jpg/ folder.

You are advised to start from the \hyperref[chapter00]{README} in the
gStore root directory, and visit other chapters only when needed. At
last, you will see all documents from link to link if you are really
interested in gStore.

\clearpage

\hyperdef{}{chapter09}{\subsection{Chapter 09: Publications}\label{chapter09}}

\hyperdef{}{publications-related-with-gstore-are-listed-here}{\paragraph{Publications related with gStore are listed here:}\label{publications-related-with-gstore-are-listed-here}}

\begin{itemize}
\item
  Lei Zou, M. Tamer $\ddot{O}$zsu,Lei Chen, Xuchuan Shen, Ruizhe Huang, Dongyan
  Zhao,
  \href{http://www.icst.pku.edu.cn/intro/leizou/projects/papers/gStoreVLDBJ.pdf}{gStore:
  A Graph-based SPARQL Query Engine}, VLDB Journal , 23(4): 565-590,
  2014.
\item
  Lei Zou, Jinghui Mo, Lei Chen,M. Tamer $\ddot{O}$zsu, Dongyan Zhao,
  \href{http://www.icst.pku.edu.cn/intro/leizou/projects/papers/p482-zou.pdf}{gStore:
  Answering SPARQL Queries Via Subgraph Matching}, Proc. VLDB 4(8):
  482-493, 2011.
\item
  Xuchuan Shen, Lei Zou, M. Tamer $\ddot{O}$zsu, Lei Chen, Youhuan Li, Shuo Han,
  Dongyan Zhao,
  \href{http://www.icst.pku.edu.cn/intro/leizou/projects/papers/demo.pdf}{A
  Graph-based RDF Triple Store}, ICDE 2015: 1508-1511.
\item
  Peng Peng, Lei Zou, M. Tamer $\ddot{O}$zsu, Lei Chen, Dongyan Zhao: \href{http://arxiv.org/pdf/1411.6763v4.pdf}{Processing
  SPARQL queries over distributed RDF graphs}. VLDB Journal 25(2): 243-268 (2016).
\item
  Dong Wang, Lei Zou, Yansong Feng, Xuchuan Shen, Jilei Tian, and
  Dongyan Zhao,
  \href{http://www.icst.pku.edu.cn/intro/leizou/projects/papers/Store.pdf}{S-store:
  An Engine for Large RDF Graph Integrating Spatial Information}, in
  Proc. 18th International Conference on Database Systems for Advanced
  Applications (DASFAA), pages 31-47, 2013.
\item
  Dong Wang, Lei Zou and Dongyan Zhao,
  \href{http://www.icst.pku.edu.cn/intro/leizou/projects/papers/edbtdemo2014.pdf}{gst-Store:
  An Engine for Large RDF Graph Integrating Spatiotemporal Information},
  in Proc. 17th International Conference on Extending Database
  Technology (EDBT), pages 652-655, 2014 (demo).
\item
  Lei Zou, Yueguo Chen,
  \href{http://www.icst.pku.edu.cn/intro/leizou/documentation/pdf/2012CCCF.pdf}{A
  Survey of Large-Scale RDF Data Management}, Comunications of CCCF
  Vol.8(11): 32-43, 2012 (Invited Paper, in Chinese).
\end{itemize}

\clearpage

\hyperdef{}{chapter10}{\subsection{Chapter 10: Limitations}\label{chapter10}}

\begin{enumerate}
\item
  Queries related with unbounded predicates are not supported.
\item
  This version only supports SPARQL select query.
\item
  Only support RDF file in N3 file format. More file formats will be
  supported in the next version.
\end{enumerate}

\clearpage

\hyperdef{}{chapter11}{\subsection{Chapter 11: Frequently Asked Questions}\label{chapter11}}

\hyperdef{}{when-i-use-the-newer-gstore-system-to-query-the-original-database-why-error}{\paragraph{When
I use the newer gStore system to query the original database, why
error?}\label{when-i-use-the-newer-gstore-system-to-query-the-original-database-why-error}}
\quad\\
The database produced by gStore contains several indexes, whose
structures may have been changed in the new gStore version. So, please
rebuild your dataset just in case.

\hyperdef{}{why-error-when-i-try-to-write-programs-based-on-gstore-just-like-the-maingconsolecpp}{\paragraph{Why
error when I try to write programs based on gStore, just like the
Main/gconsole.cpp?}\label{why-error-when-i-try-to-write-programs-based-on-gstore-just-like-the-maingconsolecpp}}
\quad\\
You need to add these phrases at the beginning of your main program,
otherwise gStore will not run correctly:\\ //NOTICE:this is needed to
set several debug files\\ Util util;

\hyperdef{}{why-does-gstore-report-garbage-collection-failed-error-when-i-use-teh-java-api}{\paragraph{\texorpdfstring{Why
does gStore report ``garbage collection failed'' error when I use the
Java
API?}{Why does gStore report garbage collection failed error when I use teh Java API?}}\label{why-does-gstore-report-garbage-collection-failed-error-when-i-use-teh-java-api}}
\quad\\
You need to adjust the parameters of jvm, see
\href{http://www.cnblogs.com/edwardlauxh/archive/2010/04/25/1918603.html}{url1}
and
\href{http://www.cnblogs.com/redcreen/archive/2011/05/04/2037057.html}{url2}
for details.

\hyperdef{}{when-i-compile-the-code-in-archlinux-why-the-error-that-no-ltermcap-is-reported}{\paragraph{\texorpdfstring{When
I compile the code in ArchLinux, why the error that ``no -ltermcap'' is
reported?}{When I compile the code in ArchLinux, why the error that no -ltermcap is reported?}}\label{when-i-compile-the-code-in-archlinux-why-the-error-that-no-ltermcap-is-reported}}
\quad\\
In ArchLinux, you only need to use \texttt{-lreadline} to link the
readline library. Please remove the \texttt{-ltermcap} in the makefile
which is located in the root of the gStore project if you would like to
use ArchLinux.

\hyperdef{}{why-does-gstore-report-errors-that-the-format-of-some-rdf-datasets-are-not-supported}{\paragraph{Why
does gStore report errors that the format of some RDF datasets are not
supported?}\label{why-does-gstore-report-errors-that-the-format-of-some-rdf-datasets-are-not-supported}}
\quad\\
gStore does not support all RDF formats currently, please see
\href{run:../../test/format_question.txt}{formats}
for details. However, it is quite easy for you to convey your RDF data format to the N3 file format that is used in gStore.

\hyperdef{}{when-i-read-on-github-why-are-some-documents-unable-to-be-opened}{\paragraph{When
I read on GitHub, why are some documents unable to be
opened?}\label{when-i-read-on-github-why-are-some-documents-unable-to-be-opened}}
\quad\\
Codes, markdowns or other text files, and pictures can be read directly
on GitHub. However, if you are using some light weight browsers like
midori, for files in pdf type, please download them and read on your
computer or other devices.

\hyperdef{}{why-sometimes-strange-characters-appear-when-i-use-gstore}{\paragraph{Why
sometimes strange characters appear when I use
gStore?}\label{why-sometimes-strange-characters-appear-when-i-use-gstore}}
\quad\\
There are some documents's names are in Chinese, and you don't need to
worry about it.

\hyperdef{}{in-centos7-if-the-watdivdba-generated-database-after-gbuild-is-copied-or-compresseduncompressed-the-size-of-watdivdb-will-be-differentgenerally-increasing-if-using-du-h-command-to-check}{\paragraph{\texorpdfstring{In
centos7, if the watdiv.db(a generated database after gbuild) is copied or
compressed/uncompressed, the size of watdiv.db will be
different(generally increasing) if using \texttt{du\ -h} command to
check?}{In centos7, if the watdiv.db(a generated database after gbuild) is copied or compressed/uncompressed, the size of watdiv.db will be different(generally increasing) if using du -h command to check?}}\label{in-centos7-if-the-watdivdba-generated-database-after-gbuild-is-copied-or-compresseduncompressed-the-size-of-watdivdb-will-be-differentgenerally-increasing-if-using-du-h-command-to-check}}
\quad\\
It's the change of B+-trees' size in watdiv/kv\_store/ that causes the
change of the whole database's size. The reason is that in
storage/Storage.cpp, many operations use fseek to move file pointer. As
everyone knows, file is organized in blocks, and if we request for new
block, file pointer may be moved beyond the end of this file(file
operations are all achieved by C in gStore, no errors are reported),
then contents will be written in the new position!

In \textbf{Advanced Programming In The Unix Environment}, ``file hole''
is used to describe this phenomenon. ``file hole'' will be filled with
0, and it's also one part of the file. You can use \texttt{ls\ -l} to
see the size of file(computing the size of holes), while \texttt{du\ -h}
command shows the size of blocks that directory/file occupies in system.
Generally, the output of \texttt{du\ -h} is large than that of
\texttt{ls\ -l}, but if ``file hole'' exists, the opposite is the case
because the size of holes are neglected.

The actual size of files containing holes are fixed, while in some
operation systems, holes will be transformed to contents(also 0) when
copied. Operation \texttt{mv} will not affect the size if not across
different devices.(only need to adjust the file tree index) However,
\texttt{cp} and all kinds of compress methods need to scan the file and
transfer data.(there are two ways to achieve \texttt{cp} command,
neglect holes or not, while the output size of \texttt{ls\ -l} not
varies)

It is valid to use ``file hole'' in C, and this is not an error, which
means you can go on using gStore. We achieve a small program to describe
the ``file holes'', you can download and try it yourself.

\hyperdef{}{in-gclient-console-a-database-is-built-queried-and-then-i-quit-the-console-next-time-i-enter-the-console-load-the-originally-imported-database-but-no-output-for-any-queriesoriginally-the-output-is-not-empty}{\paragraph{In
gclient console, a database is built, queried, and then I quit the
console. Next time I enter the console, load the originally imported
database, but no output for any queries(originally the output is not
empty)?}\label{in-gclient-console-a-database-is-built-queried-and-then-i-quit-the-console-next-time-i-enter-the-console-load-the-originally-imported-database-but-no-output-for-any-queriesoriginally-the-output-is-not-empty}}
\quad\\
You need to unload the using database before quiting the gclient
console, otherwise errors come.

\hyperdef{}{if-query-results-contain-null-value-how-can-i-use-the-fulltest-utility-tab-separated-method-will-cause-problem-here-because-null-value-cannot-be-checked}{\paragraph{\texorpdfstring{If
query results contain null value, how can I use the
\href{run:../../test/full_test.sh}{full\_test}
utility? Tab separated method will cause problem here because null value
cannot be
checked!}{If query results contain null value, how can I use the full\_test utility? Tab separated method will cause problem here because null value cannot be checked!}}\label{if-query-results-contain-null-value-how-can-i-use-the-fulltest-utility-tab-separated-method-will-cause-problem-here-because-null-value-cannot-be-checked}}
\quad\\
You may use other programming language(for example, Python) to deal with
the null value cases. For example, you can change null value in output
to special character like `,', later you can use the
\href{run:../../test/full_test.sh}{full\_test}
utility.

\hyperdef{}{when-i-compile-and-run-the-api-examples-it-reports-the-unable-to-connect-to-server-error}{\paragraph{\texorpdfstring{When
I compile and run the API examples, it reports the ``unable to connect
to server''
error?}{When I compile and run the API examples, it reports the unable to connect to server error?}}\label{when-i-compile-and-run-the-api-examples-it-reports-the-unable-to-connect-to-server-error}}
\quad\\
Please use \texttt{./gserver} command to start up a gStore server first,
and notice that the server ip and port must be matched.

\hyperdef{}{when-i-use-the-java-api-to-write-my-own-program-it-reports-not-found-main-class-error}{\paragraph{\texorpdfstring{When
I use the Java API to write my own program, it reports ``not found main
class''
error?}{When I use the Java API to write my own program, it reports not found main class error?}}\label{when-i-use-the-java-api-to-write-my-own-program-it-reports-not-found-main-class-error}}
\quad\\
Please ensure that you include the position of your own program in class
path of java. The whole command should be something like
\texttt{java\ -cp\ /home/bookug/project/devGstore/api/java/lib/GstoreJavaAPI.jar:.\ JavaAPIExample},
and the ``:.'' in this command cannot be neglected.

%\begin{center}\rule{0.5\linewidth}{\linethickness}\end{center}
\clearpage
\hyperdef{}{chapter12}{\subsection{Chapter 12: Recipe Book}\label{chapter12}}

\textbf{This chapter introduces some useful tricks if you are using
gStore to implement applications.}

\emph{no tips available now}

\clearpage

\part{Others}

\hyperdef{}{chapter13}{\subsection{Chapter 13: Contributors}\label{chapter13}}

Please contact with Lei Zou(zoulei@pku.edu.cn), Li Zeng(zengli-bookug@pku.edu.cn), Jiaqi Chen(chenjiaqi93@pku.edu.cn) and Peng Peng(pku09pp@pku.edu.cn) if you have suggestions or comments about gStore or you need help when using gStore.

\hyperdef{}{faculty}{\paragraph{Faculty}\label{faculty}}

\begin{itemize}
\item
  Lei Zou (Peking University) Project Leader
\item
  M. Tamer {\"O}zsu (University of Waterloo)
\item
  Lei Chen (Hong Kong University of Science and Technology)
\item
  Dongyan Zhao (Peking Univeristy)
\item 
  Zhiyuan Deng (Wuhan University)
\end{itemize}

\hyperdef{}{students}{\paragraph{Students}\label{students}}

\quad \\

\textit{Li Zeng and Jiaqi Chen are responsible for the gStore system optimization. Peng Peng is responsible for the distributed version of gStore, which is expected to be released before October.}

\begin{itemize}
\item
  Peng Peng (Peking University) (PhD student)
  %email: \href{mailto:pku09pp@pku.edu.cn}{pku09pp@pku.edu.cn}
\item
  Youhuan Li (Peking University) (PhD student)
  %email: \href{mailto:liyouhuan@pku.edu.cn}{liyouhuan@pku.edu.cn}
\item
  Shuo Han (Peking University) (PhD student)
  %email: \href{mailto:hanshuo@pku.edu.cn}{hanshuo@pku.edu.cn}
\item
  Li Zeng (Peking University) (Master student)
  %email: \href{mailto:zengli-syzz@pku.edu.cn}{zengli-syzz@pku.edu.cn}
\item
  Jiaqi Chen (Peking University) (Master student)
  %email: \href{mailto:chenjiaqi93@pku.edu.cn}{chenjiaqi93@pku.edu.cn}
\end{itemize}

\hyperdef{}{alumni}{\paragraph{Alumni}\label{alumni}}

\begin{itemize}
\item
  Xuchuan Shen (Peking University) (Master's student, graduated)
  %email: \href{mailto:shenxuchuan@pku.edu.cn}{shenxuchuan@pku.edu.cn}
\item
  Dong Wang (Peking University) (PhD student, graduated)
  %email: \href{mailto:wangdong@pku.edu.cn}{wangdong@pku.edu.cn}
\item
  Ruizhe Huang (Peking University) (Undergraudate intern, graduated)
\item
  Jinhui Mo (Peking University) (Master's, graduated)
\end{itemize}

\clearpage

\hyperdef{}{chapter14}{\subsection{Chapter 14: Updated Logs}\label{chapter14}}

\hyperdef{}{jan-10-2017}{\subsubsection{Jan 10,
2017}\label{jan-10-2017}}

preFilter() function in Join module is optimazed using the pre2num structure, as well as the choose\_next\_node() function.
A global string buffer is added to lower the cost of getFinalResult(), and the time of answering queries is reduced greatly.

In addition, we assign buffers of different size for all B+ trees.(some of them are more important and more frequently used)
WangLibo merges several B+ trees into one, and the num of all B+ trees are reduced to 9 from 17.
This strategy not only reduces the space cost, but also reduces the memory cost, meanwhile speeding up the build process and query process.

What is more, ChenJiaqi has done a lot of work to optimaze the SPARQL query.
For example, some unconnected SPARQL query graphs are dealed specially.

\hyperdef{}{sep-15-2016}{\subsubsection{Sep 15,
2016}\label{sep-15-2016}}

ZengLi splits the KVstore into 3 parts according to the types of key and value, i.e. int2string, string2int and string2string.
In addition, updates are supported now. 
You can insert, delete or modify some triples in the gStore database.
In fact, only insert() and remove() are implemented, while the modify() are supported by removing first and insert again.

\hyperdef{}{jun-20-2016}{\subsubsection{Jun 20,
2016}\label{jun-20-2016}}

ZengLi has enabled the gStore to answer queries with predicate variables. 
In addition, the structures of many queries have been studied to speed up the query processing.
ChenJiaqi rewrites the sparql query plan to acquire a more efficient one, which brings many benefits to us.

\hyperdef{}{apr-01-2016}{\subsubsection{Apr 01,
2016}\label{apr-01-2016}}

The structure of this project has changed a lot now. A new join method
has been achieved and we use it to replace the old one. The test result
shows that speed is improved and the memory cost is lower. We also do
some change to Parser/Sparql*, which are all generated by ANTLR. They
must be modified because the code is in C, which brings several multiple
definition problems, and its size is too large.

There is a bug in the original Stream module, which brings some control
characters to the output, such as \^{}C, \^{}V and so on. We have fixed
it now and enabled the Stream to sort the output strings(both internal
and external). In addition, SPARQL queries which are not BGP(Basic Graph
Pattern) are also supported now, using the naive method.

A powerful interactive console, which is named \texttt{gconsole} now, is
achieved to bring convenience to users. What is more, we use valgrind
tools to test our system, and deal with several memory leaks.

The docs and API have also changed, but this is of little importance.

\hyperdef{}{nov-06-2015}{\subsubsection{Nov 06,
2015}\label{nov-06-2015}}

We merge several classes(like Bstr) and adjust the project structure, as
well as the debug system.

In addition, most warnings are removed, except for warnings in Parser
module, which is due to the use of ANTLR.

What is more, we change RangeValue module to Stream, and add Stream for
ResultSet. We also better the gquery console, so now you can redirect
query results to a specified file in the gsql console.

Unable to add Stream for IDlist due to complex operations, but this is
not necessary. Realpath is used to supported soft links in the gquery
console, but it not works in Gstore.(though works if not in Gstore)

\hyperdef{}{oct-20-2015}{\subsubsection{Oct 20,
2015}\label{oct-20-2015}}

We add a gtest tool for utility, you can use it to query several
datasets with their own queries.

In addition, gquery console is improved. Readline lib is used for input
instead of fgets, and the gquery console can support commands history,
modifying command and commands completion now.

What is more, we found and fix a bug in Database/(a pointer for
debugging log is not set to NULL after fclose operation, so if you close
one database and open another, the system will fail entirely because the
system think that the debugging log is still open)

\hyperdef{}{sep-25-2015}{\subsubsection{Sep 25,
2015}\label{sep-25-2015}}

We implement the version of B+Tree, and replace the old one.

After testing on DBpedia, LUBM, and WatDiv benchmark, we conclude that
the new BTree performs more efficient than the old version. For the
same triple file, the new version spends shorter time on executing gload
command.

Besides, the new version can handle the long literal objects
efficiently, while triples whose object's length exceeds 4096 bytes
result in frequent inefficient split operations on the old version
BTree.

\hyperdef{}{feb-2-2015}{\subsubsection{Feb 2, 2015}\label{feb-2-2015}}

We modify the RDF parser and SPARQL parser.

Under the new RDF parser, we also redesign the encode strategy, which
reduces RDF file scanning times.

Now we can parse the standard SPARQL v1.1 grammar correctly, and can
support basic graph pattern(BGP) SPARQL queries written by this standard
grammar.

\hyperdef{}{dec-11-2014}{\subsubsection{Dec 11,
2014}\label{dec-11-2014}}

We add API for C/CPP and JAVA.

\hyperdef{}{nov-20-2014}{\subsubsection{Nov 20,
2014}\label{nov-20-2014}}

We share our gStore2.0 code as an open-source project under BSD license
on github.

\clearpage

\hyperdef{}{chapter15}{\subsection{Chapter 15: Test Result}\label{chapter15}}

\hyperdef{}{preparation}{\subsubsection{Preparation}\label{preparation}}

We have compared the performance of gStore with several other database
systems, such as \href{http://jena.apache.org/}{Jena},
\href{http://www.rdf4j.org/}{Sesame},
\href{http://virtuoso.openlinksw.com/}{Virtuoso} and so on. Contents to
be compared are the time to build database, the size of the built
database, the time to answer single SPARQL query and the matching case
of single query's results. In addition, if the memory cost is very
large(\textgreater{}20G), we will record the memory cost when running
these database systems.(not accurate, just for your reference) \\

To ensure all database systems can run correctly on all datasets and
queries, the format of datasets must be supported by all database
systems and the queries should not contain update operations, aggregate
operations and operations related with uncertain predicates. Notice that
when measuring the time to answer queries, the time of loading database
index should not be included. To ensure this principle, we load the
database index first for some database systems, and warm up several
times for others. \\

Datasets used here are WatDiv, Lubm, Bsbm and DBpedia. Some of them are
provided by websites, and others are generated by algorithms. Queries
are generated by algorithms or written by us. Table \ref{table:datasets} summarizes the statistics of these datasets.

The experiment environment is a CentOS server, whose memory size is 82G
and disk size is 7T. We use
\href{run:../../test/full_test.sh}{full\_test}
to do this test.

\begin{table}[b]
	\small
	\centering
	%\vspace{-0.1in}
	\caption{Datasets}
	\begin{tabular}{|c|c|r|r|r|}
		\hline
		Dataset& Number of Triples& RDF N3 File Size(B)  & Number of Entities\\
		\hline  
		\hline
		%WatDiv 10M & 109,164,587 & 1542624409 & 0 \\
		%\hline
		%WatDiv 100M & 108,997,714 & 15,599,074,048 & 5,212,745 \\
		%\hline
		WatDiv 300M & 329,539,576  & 47,670,221,085 & 15,636,385 \\
		\hline
		%LUBM 500 & 6652613 & 801112089 & 1648692 \\
		%\hline
		LUBM 5000 & 66718642 & 8134671485 & 16437950 \\
		\hline
		DBpedia 2014 & 170784508 & 23844158944 & 7123915 \\
		\hline
		Bsbm 10000 & 34872182 & 912646084  & 526590 \\
		\hline
	\end{tabular}
	%  \vspace{-0.1in}
	\label{table:datasets}
\end{table}

%BETTER:using bsbm_100000?

\hyperdef{}{result}{\subsubsection{Result}\label{result}}

\begin{comment}
Table \ref{table:loading} shows the index size and loading time of the datasets
for different systems.

\begin{table}[htcp]
\small
\begin{threeparttable}
\begin{tabular}{|c||c|c|c||c|c|c|}
\hline
 & \multicolumn{3}{c||}{Index Size(KB)}& \multicolumn{3}{c|}{Loading Time(ms)}\\
\hline
\hline
 Datasets & gStore & Jena& Virtuoso& gStore & Jena& Virtuoso\\
  \hline
  DBpedia 2014 & 42,415,852&	23,151,272 &	-\tnote{$1$} & 8,639,666	&15,555,000	& -	\\
  \hline
  Bsbm 10000 & 1,814,480 & 718,024 & 2,080,000 & 244,153 & 76,000 & 59999  \\
  \hline
  LUBM 500  &2,171,084 &1,022,528	&	38,000,000 &	291,382&	94,000 &100,532	 \\
  \hline
  %LUBM 5000 & 23,397,548&	10,262,524	 &	- & 3,767,764	&1,098,000  &	- \\
   %\hline
  %WatDiv 10M & 2,563,168&	1,315,764	 &	10,320,000 & 532,542	&304,000	&225,464 \\
  %\hline
  WatDiv 100M & 26,566,780&	13,286,608	 &	8,615,100 & 7,879,602	&20,969,000	&16,981,470 \\
  \hline
  %WatDiv 300M & 80,166,500&	38,108,940	 &	- & 19,864,431	&25,041,000 &	- \\
  %\hline
  \end{tabular}
    \begin{tablenotes}
  \small
  \item[$1$] ``-'' means that loading does not terminate in 10 hour
  \end{tablenotes}
  \end{threeparttable}
  \caption{Offline Performance}
  \label{table:loading}
\end{table}
\end{comment}

The performance of different database management systems is shown in Figures \ref{fig:dbpedia2014Performance}, \ref{fig:Bsbm10000Performance}, \ref{fig:LUBMPerformance} and \ref{fig:WatDivPerformance}.

Notice that Sesame and Virtuoso are unable to operate on DBpedia 2014 and
WatDiv 300M, because the size is too large. In addition, we do not use
Sesame and Virtuoso to test on the LUBM 5000 due to format questions. 
Generally speaking, Virtuoso is not scalable, and Sesame is so weak. \\

\begin{figure}[b]%
		\resizebox{0.48\columnwidth}{!}{
				\begin{tikzpicture}[font=\large]
 		 \begin{semilogyaxis}[
               width = 12cm,
               height = 8cm,
    			ybar,
    			%ymin = 1,
    			%ymax = 5000000000,
%    ytick = {1,10,100,1000,10000,100000,1000000,10000000},
   			ymajorgrids = true,
   			ylabel = {Query Response Time (in ms)},
                %xlabel = {Queries},
    			symbolic x coords = {$Q_0$,$Q_1$,$Q_2$,$Q_3$,$Q_4$,$Q_5$,$Q_6$},
    			scaled y ticks = true,
			legend pos= north west,
 legend cell align=left
   		]
   \addplot coordinates {($Q_0$, 578) ($Q_1$, 477) ($Q_2$, 285) ($Q_3$, 465076) ($Q_4$, 127530) ($Q_5$, 1929) ($Q_9$, 447741)};

   \addplot coordinates {($Q_0$, 490) ($Q_1$, 165) ($Q_2$, 166) ($Q_3$, 3727) ($Q_4$, 3847) ($Q_5$, 393) ($Q_9$, 1309775)};

   \addplot coordinates {($Q_0$, 50) ($Q_1$, 217) ($Q_2$, 210) ($Q_3$, 23797) ($Q_4$, 5536) ($Q_5$, 2736) ($Q_9$, 9515231)};

		
   		 \legend{gStore,Jena,Virtuoso}
  		\end{semilogyaxis}
\end{tikzpicture}

		}
 \caption{Query Performance over DBpedia 2014}%
 \label{fig:dbpedia2014Performance}
\end{figure}

\begin{figure}%
		\resizebox{0.8\columnwidth}{!}{
				\begin{tikzpicture}[font=\Large]
 		 \begin{semilogyaxis}[
                width = 18cm,
               height = 9cm ,
               ymax=200000,
    			ybar,
   			ymajorgrids = true,
   			ylabel = {Query Response Time (in ms)},
                %xlabel = {Queries},
    			symbolic x coords = {$Q_{1}$,$Q_{2}$,$Q_{3}$,$Q_{4}$,$Q_{5}$,$Q_{6}$,$Q_{7}$,$Q_{8}$,$Q_{9}$,$Q_{10}$},
    bar width=5pt,
             enlarge x limits=0.07,
    			scaled y ticks = true,
			legend pos= north west,
 legend cell align=left
   		]
   \addplot coordinates {($Q_{1}$, 122) ($Q_{2}$, 121) ($Q_{3}$, 1989) ($Q_{4}$, 41) ($Q_{5}$, 37) ($Q_{6}$, 102) ($Q_{7}$, 36) ($Q_{8}$, 55) ($Q_{9}$, 39) ($Q_{10}$, 48) };


\addplot coordinates {($Q_{1}$, 308) ($Q_{2}$, 943) ($Q_{3}$, 15805) ($Q_{4}$, 375) ($Q_{5}$, 327) ($Q_{6}$, 346) ($Q_{7}$, 202) ($Q_{8}$, 322) ($Q_{9}$, 162) ($Q_{10}$, 230)};


\addplot coordinates {($Q_{1}$, 715) ($Q_{2}$, 1059) ($Q_{3}$, 8987) ($Q_{4}$, 57) ($Q_{5}$, 73) ($Q_{6}$, 283) ($Q_{7}$, 5) ($Q_{8}$, 8) ($Q_{9}$, 4) ($Q_{10}$, 242) };

\addplot coordinates {($Q_{1}$, 368) ($Q_{2}$, 1396) ($Q_{3}$, 8312) ($Q_{4}$, 148) ($Q_{5}$, 114) ($Q_{6}$, 154) ($Q_{7}$, 66) ($Q_{8}$, 160) ($Q_{9}$, 74) ($Q_{10}$, 95) };

		
   		 \legend{gStore,Jena,Virtuoso,Sesame}
  		\end{semilogyaxis}
\end{tikzpicture}

		}
 \caption{Query Performance over Bsbm 10000}%
 \label{fig:Bsbm10000Performance}
\end{figure}

\begin{figure}[h]%
   %\subfigure[LUBM 500]{%
		%\resizebox{0.98\columnwidth}{!}{
			%	\begin{tikzpicture}[font=\Large]
 		 \begin{semilogyaxis}[
               width = 25cm,
               height = 10cm,
               ymax = 100000,
    			ybar,
   			ymajorgrids = true,
   			ylabel = {Query Response Time (in ms)},
                %xlabel = {Queries},
    			symbolic x coords = {$Q_{0}$,$Q_{1}$,$Q_{2}$,$Q_{3}$,$Q_{4}$,$Q_{5}$,$Q_{6}$,$Q_{7}$,$Q_{8}$,$Q_{9}$,$Q_{10}$,$Q_{11}$,$Q_{12}$,$Q_{13}$,$Q_{14}$,$Q_{15}$,$Q_{16}$,$Q_{17}$,$Q_{18}$,$Q_{19}$,$Q_{20}$,$Q_{21}$},
    bar width=5pt,
             enlarge x limits=0.02,
    			scaled y ticks = true,
			legend pos= north west,
 legend cell align=left
   		]
   \addplot coordinates {($Q_{0}$, 201) ($Q_{1}$, 204) ($Q_{2}$, 6688) ($Q_{3}$, 102) ($Q_{4}$, 98) ($Q_{5}$, 39) ($Q_{6}$, 67) ($Q_{7}$, 2327) ($Q_{8}$, 52) ($Q_{9}$, 202) ($Q_{10}$, 58) ($Q_{11}$, 98) ($Q_{12}$, 130) ($Q_{13}$, 3506) ($Q_{14}$, 53) ($Q_{15}$, 1379) ($Q_{16}$, 5049) ($Q_{17}$, 52) ($Q_{18}$, 37) ($Q_{19}$, 130) ($Q_{20}$, 36) ($Q_{21}$, 3396)};


\addplot coordinates {($Q_{0}$, 175) ($Q_{1}$, 2259) ($Q_{2}$, 1204) ($Q_{3}$, 169) ($Q_{4}$, 204) ($Q_{5}$, 172) ($Q_{6}$, 251) ($Q_{7}$, 2364) ($Q_{8}$, 718) ($Q_{9}$, 2253) ($Q_{10}$, 1308) ($Q_{11}$, 326) ($Q_{12}$, 248) ($Q_{13}$, 3125) ($Q_{14}$, 181) ($Q_{15}$, 3954) ($Q_{16}$, 3141) ($Q_{17}$, 748) ($Q_{18}$, 373) ($Q_{19}$, 298) ($Q_{20}$, 160) ($Q_{21}$, 3105)};

\addplot coordinates {($Q_{0}$, 146) ($Q_{1}$, 1890) ($Q_{2}$, 2409) ($Q_{3}$, 230) ($Q_{4}$, 26) ($Q_{5}$, 79) ($Q_{6}$, 85) ($Q_{7}$, 3800) ($Q_{8}$, 21) ($Q_{9}$, 101) ($Q_{10}$, 780) ($Q_{11}$, 39) ($Q_{12}$, 1020) ($Q_{13}$, 2877) ($Q_{14}$, 34) ($Q_{15}$, 2090) ($Q_{16}$, 6954) ($Q_{17}$, 230) ($Q_{18}$, 890) ($Q_{19}$, 370) ($Q_{20}$, 200) ($Q_{21}$, 4300)};

		
   		 \legend{gStore,Jena,Virtuoso}
  		\end{semilogyaxis}
\end{tikzpicture}

		%}
       %\label{fig:LUBM500Performance}%
       %}
       %\\
\subfigure[LUBM 5000]{%
		\resizebox{0.98\columnwidth}{!}{
				\begin{tikzpicture}[font=\Large]
 		 \begin{semilogyaxis}[
               width = 25cm,
               height = 10cm ,
    			ybar,
   			ymajorgrids = true,
   			ylabel = {Query Response Time (in ms)},
    			xlabel = {Queries},
    			symbolic x coords = {$Q_{0}$,$Q_{1}$,$Q_{2}$,$Q_{3}$,$Q_{4}$,$Q_{5}$,$Q_{6}$,$Q_{7}$,$Q_{8}$,$Q_{9}$,$Q_{10}$,$Q_{11}$,$Q_{12}$,$Q_{13}$,$Q_{14}$,$Q_{15}$,$Q_{16}$,$Q_{17}$,$Q_{18}$,$Q_{19}$,$Q_{20}$,$Q_{21}$},
    bar width=5pt,
             enlarge x limits=0.02,
    			scaled y ticks = true,
			legend pos= north west,
 legend cell align=left
   		]
   \addplot coordinates {($Q_{0}$, 1680) ($Q_{1}$, 1913) ($Q_{2}$, 69101) ($Q_{3}$, 1162) ($Q_{4}$, 1038) ($Q_{5}$, 110) ($Q_{6}$, 220) ($Q_{7}$, 23565) ($Q_{8}$, 398) ($Q_{9}$, 2000) ($Q_{10}$, 317) ($Q_{11}$, 1046) ($Q_{12}$, 483) ($Q_{13}$, 37581) ($Q_{14}$, 503) ($Q_{15}$, 4014) ($Q_{16}$, 89424) ($Q_{17}$, 381) ($Q_{18}$, 88) ($Q_{19}$, 304) ($Q_{20}$, 52) ($Q_{21}$, 42040)};


\addplot coordinates {($Q_{0}$, 262) ($Q_{1}$, 20865) ($Q_{2}$, 6814) ($Q_{3}$, 157) ($Q_{4}$, 212) ($Q_{5}$, 189) ($Q_{6}$, 257) ($Q_{7}$, 18469) ($Q_{8}$, 4365) ($Q_{9}$, 18933) ($Q_{10}$, 11305) ($Q_{11}$, 718) ($Q_{12}$, 237) ($Q_{13}$, 27059) ($Q_{14}$, 176) ($Q_{15}$, 34406) ($Q_{16}$, 26985) ($Q_{17}$, 4267) ($Q_{18}$, 845) ($Q_{19}$, 429) ($Q_{20}$, 162) ($Q_{21}$, 26853)};
		
   		 \legend{gStore,Jena}
  		\end{semilogyaxis}
\end{tikzpicture}

		}
	   \label{fig:LUBM5000Performance}%
	   }%
 \caption{Query Performance over LUBM}%
 \label{fig:LUBMPerformance}
\end{figure}

\begin{figure}[h]%
 %\subfigure[WatDiv 10M]{%
		%\resizebox{0.8\columnwidth}{!}{
			%	\begin{tikzpicture}[font=\Large]
 		 \begin{semilogyaxis}[
                width = 20cm,
               height = 9cm ,
               ymax=200000,
    			ybar,
   			ymajorgrids = true,
   			ylabel = {Query Response Time (in ms)},
                %xlabel = {Queries},
    			symbolic x coords = {$C_{1}$,$C_{2}$,$C_{3}$,$F_{1}$,$F_{2}$,$F_{3}$,$L_{1}$,$L_{2}$,$L_{3}$,$S_{1}$,$S_{2}$,$S_{3}$},
    bar width=5pt,
             enlarge x limits=0.04,
    			scaled y ticks = true,
			legend pos= north west,
 legend cell align=left
   		]
   \addplot coordinates {($C_{1}$, 122) ($C_{2}$, 121) ($C_{3}$, 1989) ($F_{1}$, 41) ($F_{2}$, 37) ($F_{3}$, 102) ($L_{1}$, 36) ($L_{2}$, 55) ($L_{3}$, 39) ($S_{1}$, 48) ($S_{2}$, 53) ($S_{3}$, 40)};


\addplot coordinates {($C_{1}$, 308) ($C_{2}$, 943) ($C_{3}$, 15805) ($F_{1}$, 375) ($F_{2}$, 327) ($F_{3}$, 346) ($L_{1}$, 202) ($L_{2}$, 322) ($L_{3}$, 162) ($S_{1}$, 230) ($S_{2}$, 283) ($S_{3}$, 344)};


\addplot coordinates {($C_{1}$, 715) ($C_{2}$, 1059) ($C_{3}$, 8987) ($F_{1}$, 57) ($F_{2}$, 73) ($F_{3}$, 283) ($L_{1}$, 5) ($L_{2}$, 8) ($L_{3}$, 4) ($S_{1}$, 242) ($S_{2}$, 7) ($S_{3}$, 50)};

\addplot coordinates {($C_{1}$, 368) ($C_{2}$, 1396) ($C_{3}$, 8312) ($F_{1}$, 148) ($F_{2}$, 114) ($F_{3}$, 154) ($L_{1}$, 66) ($L_{2}$, 160) ($L_{3}$, 74) ($S_{1}$, 95) ($S_{2}$, 143) ($S_{3}$, 279)};

		
   		 \legend{gStore,Jena,Virtuoso,Sesame}
  		\end{semilogyaxis}
\end{tikzpicture}

		%}
       %\label{fig:WatDiv10MPerformance}%
       %}
   %\subfigure[WatDiv 100M]{%
		%\resizebox{0.8\columnwidth}{!}{
			%	\begin{tikzpicture}[font=\Large]
 		 \begin{semilogyaxis}[
                width = 20cm,
               height = 9cm ,
    			ybar,
   			ymajorgrids = true,
   			ylabel = {Query Response Time (in ms)},
    			xlabel = {Queries},
    			symbolic x coords = {$C_{1}$,$C_{2}$,$C_{3}$,$F_{1}$,$F_{2}$,$F_{3}$,$L_{1}$,$L_{2}$,$L_{3}$,$S_{1}$,$S_{2}$,$S_{3}$},
    bar width=5pt,
             enlarge x limits=0.04,
    			scaled y ticks = true,
			legend pos= north west,
 legend cell align=left
   		]
   \addplot coordinates {($C_{1}$, 1408) ($C_{2}$, 1252) ($C_{3}$, 28677) ($F_{1}$, 126) ($F_{2}$, 215) ($F_{3}$, 971) ($L_{1}$, 129) ($L_{2}$, 623) ($L_{3}$, 321) ($S_{1}$, 135) ($S_{2}$, 253) ($S_{3}$, 41)};


\addplot coordinates {($C_{1}$, 1220) ($C_{2}$, 4617) ($C_{3}$, 153329) ($F_{1}$, 978) ($F_{2}$, 346) ($F_{3}$, 491) ($L_{1}$, 198) ($L_{2}$, 895) ($L_{3}$, 232) ($S_{1}$, 190) ($S_{2}$, 424) ($S_{3}$, 882)};


\addplot coordinates {($C_{1}$, 1087245) ($C_{2}$, 115820) ($C_{3}$, 74156) ($F_{1}$, 96) ($F_{2}$, 350) ($F_{3}$, 741) ($L_{1}$, 43) ($L_{2}$, 67) ($L_{3}$, 3) ($S_{1}$, 656) ($S_{2}$, 47) ($S_{3}$, 44)};

		
   		 \legend{gStore,Jena,Virtuoso}
  		\end{semilogyaxis}
\end{tikzpicture}

		%}
       %\label{fig:WatDiv100MPerformance}%
       %}
\subfigure[WatDiv 300M]{%
		\resizebox{0.8\columnwidth}{!}{
				\begin{tikzpicture}[font=\Large]
 		 \begin{semilogyaxis}[
                width = 20cm,
               height = 9cm ,
    			ybar,
   			ymajorgrids = true,
   			ylabel = {Query Response Time (in ms)},
                %xlabel = {Queries},
    			symbolic x coords = {$C_{1}$,$C_{2}$,$C_{3}$,$F_{1}$,$F_{2}$,$F_{3}$,$L_{1}$,$L_{2}$,$L_{3}$,$S_{1}$,$S_{2}$,$S_{3}$},
    bar width=5pt,
             enlarge x limits=0.04,
    			scaled y ticks = true,
			legend pos= north west,
 legend cell align=left
   		]
   \addplot coordinates {($C_{1}$, 19602) ($C_{2}$, 11627) ($C_{3}$, 230831) ($F_{1}$, 1044) ($F_{2}$, 6194) ($F_{3}$, 134134) ($L_{1}$, 399) ($L_{2}$, 12548) ($L_{3}$, 620) ($S_{1}$, 490) ($S_{2}$, 2858) ($S_{3}$, 190)};


\addplot coordinates {($C_{1}$, 12241) ($C_{2}$, 14458) ($C_{3}$, 497314) ($F_{1}$, 1679) ($F_{2}$, 2844) ($F_{3}$, 58072) ($L_{1}$, 267) ($L_{2}$, 11884) ($L_{3}$, 711) ($S_{1}$, 359) ($S_{2}$, 1840) ($S_{3}$, 750)};

		
   		 \legend{gStore,Jena,Virtuoso}
  		\end{semilogyaxis}
\end{tikzpicture}

		}
	   \label{fig:WatDiv300MPerformance}%
	   }%
 \caption{Query Performance over WatDiv}%
 \label{fig:WatDivPerformance}
\end{figure}


This program produces many logs placed in result.log/, load.log/ and
time.log/. You can see that all results of all queries are matched by
viewing files in result.log/, and the time cost and space cost of gStore
to build database are larger than others by viewing files in load.log/.
More precisely, there is an order of magnitude difference between gStore
and others in the time/space cost of building database.

Through analysing time.log/, we can find that gStore behave better than
others on very complicated queries(many variables, circles, etc). For
other simple queries, there is not much difference between the time of
these database systems.

Generally speaking, the memory cost of gStore when answering queries is
higher than others. More complicated the query is and more large the
dataset is, more apparent the phenomenon is.

You can find more detailed information in \href{run:../pdf/gstore_test_report.pdf}{original test report}. Notice that some questions in the test report have already be solved now.
The latest test report is \href{run:../test/formal_experiment.pdf}{formal experiment}.

\clearpage

\hyperdef{}{chapter16}{\subsection{Chapter 16: Future Plan}\label{chapter16}}

\hyperdef{}{improve-the-core}{\subsubsection{Improve The
Core}\label{improve-the-core}}

\begin{itemize}
\item
  optimize the join operation of node candidates. multiple methods
  should be achieved, and design a score module to select a best one
\item
  add numeric value query function. need to answer numeric range query
  efficiently and space consume cannot be too large
\item
  add a control module to heuristically select an kind of index for a
  SPARQL query to filter(not always vstree)
\item
  typedef all frequently used types, to avoid inconsistence and high
  modify cost
\end{itemize}

\hyperdef{}{better-the-interface}{\subsubsection{Better The
Interface}\label{better-the-interface}}

\begin{itemize}
\item
  build a console named gconsole, which provides all operations
  supported by gStore.(parser and auto-complete is required)
\item
  write web interface for gStore, and a web page to operate on it, just
  like virtuoso
\end{itemize}

\hyperdef{}{idea-collection-box}{\subsubsection{Idea Collection
Box}\label{idea-collection-box}}

\begin{itemize}
\item
  to support soft links in console: realpath not work\ldots{}(redefined
  in ANTLR?)
\item
  store command history for consoles
\item
  warnings remain in using Parser/(antlr)!(modify sparql.g 1.1 and
  regenerate). change name to avoid redefine problem, or go to use
  executable to parse
\item
  build compress module(such as key-value module and stream module), but
  the latter just needs one-pass read/write, which may causes the
  compress method to be used both in disk and memory. all operations of
  string in memory can be changed to operations after compress: provide
  compress/archive interface, compare function. there are many compress
  algorithms to be chosen, then how to choose? what about utf-8 encoding
  problem? this method can lower the consume of memory and disk, but
  consumes more CPU. However, the time is decided by isomorphism. Simple
  compress is not good, but too complicated method will consume too much
  time, how to balance? (merge the continuous same characters, Huffman
  tree)
\item
  mmap to speedup KVstore?
\item
  the strategy for Stream:is 85\% valid? consider sampling, analyse the
  size of result set and decide strategy? how to support order by: sort
  in memory if not put in file; otherwise, partial sort in memory, then
  put into file, then proceed external sorting
\end{itemize}

\clearpage

\hyperdef{}{chapter17}{\subsection{Chapter 17: Thanks List}\label{chapter17}}

\textit{This chapter lists people who inspire us or contribute to this project.}

\paragraph{GitHub user zhangxiaoyang \\
	https://github.com/zhangxiaoyang \\
	1. add python api \\
	2. fix logger message}

%\begin{center}\rule{0.5\linewidth}{\linethickness}\end{center}
\clearpage

\hyperdef{}{chapter18}{\subsection{Chapter 18: Legal Issues}\label{chapter18}}

%\textbf{We are trying our best to avoid errors. However, if you encounter any unrecovable disaster when using this system, we shall not be responsible for it.}

%below is the BSD LICENSE: http://baike.baidu.com/link?url=a7XUsshp1Sd_DvF7oIJ_CpHTOZryu4ACSSj1AyQl1GU9XL5pPEj9RxIEMF1nC213VvJ2quhWTK9OCZot-CS0LK
%The following is a BSD license template. To generate your own license, change the values of OWNER, ORGANIZATION and YEAR from their original values as given here, and substitute your own.
%Note: The advertising clause in the license appearing on BSD Unix files was officially rescinded by the Director of the Office of Technology Licensing of the University of California on July 22 1999. He states that clause 3 is "hereby deleted in its entirety."
%Note the new BSD license is thus equivalent to the MIT License, except for the no-endorsement final clause.
%<OWNER> = gStore team
%<ORGANIZATION> = Peking University
%<YEAR> = 2016
%In the original BSD license, both occurrences of the phrase "COPYRIGHT HOLDERS AND CONTRIBUTORS" in the disclaimer read "REGENTS AND CONTRIBUTORS".
%Here is the license template:
%Copyright (c) &lt;YEAR&gt;, &lt;OWNER&gt;

Copyright (c) 2016 gStore team \\
All rights reserved. \\

Redistribution and use in source and binary forms, with or without modification, are permitted provided that the following conditions are met:

Redistributions of source code must retain the above copyright notice, this list of conditions and the following disclaimer.

Redistributions in binary form must reproduce the above copyright notice, this list of conditions and the following disclaimer in the documentation and/or other materials provided with the distribution.

Neither the name of the Peking University nor the names of its contributors may be used to endorse or promote products derived from this software without specific prior written permission. \\

THIS SOFTWARE IS PROVIDED BY THE COPYRIGHT HOLDERS AND CONTRIBUTORS "AS IS" AND ANY EXPRESS OR IMPLIED WARRANTIES, INCLUDING, BUT NOT LIMITED TO, THE IMPLIED WARRANTIES OF MERCHANTABILITY AND FITNESS FOR A PARTICULAR PURPOSE ARE DISCLAIMED. IN NO EVENT SHALL THE COPYRIGHT OWNER OR CONTRIBUTORS BE LIABLE FOR ANY DIRECT, INDIRECT, INCIDENTAL, SPECIAL, EXEMPLARY, OR CONSEQUENTIAL DAMAGES (INCLUDING, BUT NOT LIMITED TO, PROCUREMENT OF SUBSTITUTE GOODS OR SERVICES; LOSS OF USE, DATA, OR PROFITS; OR BUSINESS INTERRUPTION) HOWEVER CAUSED AND ON ANY THEORY OF LIABILITY, WHETHER IN CONTRACT, STRICT LIABILITY, OR TORT (INCLUDING NEGLIGENCE OR OTHERWISE) ARISING IN ANY WAY OUT OF THE USE OF THIS SOFTWARE, EVEN IF ADVISED OF THE POSSIBILITY OF SUCH DAMAGE. \\

What's more, you need to include the label "powered by gStore", as well as the logo of gStore, in your software product which is using gStore.

We would be very grateful if you are willing to tell us about your name, institution, purpose and email.  Such information can be sent to us by emailing to \href{mailto:gStoreDB@gmail.com}{gStoreDB@gmail.com}, and we promise not to reveal privacy.
%using gmail or website

\clearpage

\section{End}
\textbf{Thank you for reading this document. If any question or advice, or you have interests in this project, please don't hesitate to get in touch with us.}

\clearpage

\end{document}

